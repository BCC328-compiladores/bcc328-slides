% Intended LaTeX compiler: pdflatex
\documentclass[11pt]{article}
\usepackage[utf8]{inputenc}
\usepackage[T1]{fontenc}
\usepackage{graphicx}
\usepackage{longtable}
\usepackage{wrapfig}
\usepackage{rotating}
\usepackage[normalem]{ulem}
\usepackage{amsmath}
\usepackage{amssymb}
\usepackage{capt-of}
\usepackage{hyperref}
\author{Construção de compiladores I}
\date{}
\title{Revisão de Haskell}
\hypersetup{
 pdfauthor={Construção de compiladores I},
 pdftitle={Revisão de Haskell},
 pdfkeywords={},
 pdfsubject={},
 pdfcreator={Emacs 28.2 (Org mode 9.7)}, 
 pdflang={English}}
\begin{document}

\maketitle
\section*{Objetivos}
\label{sec:orgbbdbe53}

\subsection*{Objetivos}
\label{sec:org6f1efe3}

\begin{itemize}
\item Revisar conceitos sobre mônadas e separação entre funções puras e
impuras em Haskell.

\item Finalização da implementação do compilador.
\end{itemize}
\section*{Markdown}
\label{sec:org3d9c5b5}

\subsection*{Markdown}
\label{sec:org9aefd59}

\begin{itemize}
\item Iniciamos nosso curso com o projeto de um mini-compilador de Markdown.

\item Objetivo: revisar conceitos da linguagem Haskell
\end{itemize}
\subsection*{Markdown}
\label{sec:orgcdab37e}

\begin{itemize}
\item Até o momento:
\begin{itemize}
\item Fazemos o parsing de markdown para uma AST.
\item Definimos uma EDSL para gerar HTML.
\item Transformamos a AST em HTML usando a EDSL.
\end{itemize}

\item O que falta?
\end{itemize}
\subsection*{Markdown}
\label{sec:org3b8cd02}

\begin{itemize}
\item Interação com usuário
\begin{itemize}
\item Receber nome de arquivos de entrada / saída.
\item Ler o arquivo de entrada.
\item Escrever o arquivo de resultado.
\end{itemize}
\end{itemize}
\subsection*{Markdown}
\label{sec:orgf5d878f}

\begin{itemize}
\item I/O é uma forma de efeito colateral
\begin{itemize}
\item Resultado não depende apenas dos argumentos de entrada do código.
\end{itemize}

\item A priori, Haskell impõe que todas as funções devem ser \textbf{puras}.
\begin{itemize}
\item Pura = sem efeitos colaterais
\end{itemize}
\end{itemize}
\subsection*{Markdown}
\label{sec:org0c1bcb2}

\begin{itemize}
\item Mas como fazer I/O ou outro tipo de efeito colateral em Haskell?
\begin{itemize}
\item Devemos utilizar \textbf{mônadas}.
\end{itemize}
\end{itemize}
\section*{Mônadas}
\label{sec:org61f7d6b}

\subsection*{Mônadas}
\label{sec:org4e4e3f5}

\begin{itemize}
\item Haskell é uma linguagem puramente funcional.

\item Por padrão, todas as funções em Haskell são puras
\begin{itemize}
\item Não possuem efeitos colaterais: I/O, estado, exceções.
\end{itemize}
\end{itemize}
\subsection*{Mônadas}
\label{sec:orgfdd5163}

\begin{itemize}
\item Funções puras são garantidas de retornar o mesmo resultado para mesmas entradas.
\begin{itemize}
\item Facilita o teste.
\item Facilita o raciocínio formal.
\end{itemize}
\end{itemize}
\subsection*{Mônadas}
\label{sec:org1cf0149}

\begin{itemize}
\item Porém, para um compilador, precisamos de entrada e saída.
\end{itemize}
\subsection*{Mônadas}
\label{sec:orgd8dea09}

\begin{itemize}
\item Para lidar com efeitos colaterais, Haskell utiliza o conceito de mônada.
\begin{itemize}
\item Mônadas são tipos que implementam a seguinte classe:
\end{itemize}
\end{itemize}

\begin{verbatim}
class Applicative m => Monad m where
   return :: a -> m a
   (>>=)  :: m a -> (a -> m b) -> m b
\end{verbatim}
\subsection*{Mônadas}
\label{sec:org78463e6}

\begin{itemize}
\item O operador \texttt{>{}>{}=} permite compor sequencialmente computações em uma mônada.

\item Exemplo: ler um nome e imprimir uma mensagem de saudação.
\end{itemize}

\begin{verbatim}
ex :: IO ()
ex = getLine >>= \ s -> putStrLn $ "Hello " ++ s
\end{verbatim}
\subsection*{Mônadas}
\label{sec:org4f5e5ea}

\begin{verbatim}
getLine >>= \ s -> putStrLn $ "Hello " ++ s
\end{verbatim}

\begin{itemize}
\item Entendendo o código:
\begin{itemize}
\item Primeiro usamos \texttt{getLine} para ler uma string do console.
\item O resultado é passado para a função
\end{itemize}
\end{itemize}

\begin{verbatim}
\ s -> putStrLn $ "Hello " ++ s
\end{verbatim}
\subsection*{Mônadas}
\label{sec:org8fddda5}

\begin{itemize}
\item Uso de \texttt{>{}>{}=} prejudica a legibilidade do código.

\item Solução: notação \textbf{do}
\end{itemize}
\subsection*{Mônadas}
\label{sec:org47ae72e}

\begin{itemize}
\item A notação \texttt{do} pode simplificar computações em mônadas.
\begin{itemize}
\item Compilador traduz o \texttt{do} para usos de \texttt{>{}>{}=}.
\end{itemize}
\end{itemize}

\begin{verbatim}
ex1 :: IO ()
ex1 = do
     s <- getLine
     putStrLn $ "Hello " ++ s
\end{verbatim}
\subsection*{Mônadas}
\label{sec:orga600002}

\begin{itemize}
\item Então, utilizaremos mônadas apenas para realizar I/O?
\begin{itemize}
\item Sim!
\end{itemize}
\item Toda a lógica será implementada por funções \textbf{puras}.
\begin{itemize}
\item Não dependem de fatores externos ao programa.
\end{itemize}
\end{itemize}
\subsection*{Mônadas}
\label{sec:org2f62bdf}

\begin{itemize}
\item Padrão: Functional core / imperative shell
\begin{itemize}
\item Funcionalidade central implementada por funções puras.
\item Código puro ``envolvido por uma casca'' de funções em mônadas
\end{itemize}
\end{itemize}
\subsection*{Mônadas}
\label{sec:org7b5146b}

\begin{itemize}
\item Para leitura / escrita utilizaremos a biblioteca \texttt{System.IO}

\item Tipos utilizados:
\end{itemize}

\begin{verbatim}
  FilePath
  Handle
  data IOMode
     = ReadMode
     | WriteMode
     | AppendMode
     | ReadWriteMode
\end{verbatim}
\subsection*{Mônadas}
\label{sec:orgf70ee6d}

\begin{itemize}
\item Para leitura / escrita utilizaremos a biblioteca \texttt{System.IO}

\item Funções utilizadas:
\end{itemize}

\begin{verbatim}
openFile :: FilePath -> IOMode -> IO Handle
hClose :: Handle -> IO ()
stdin :: Handle
stdout :: Handle
hGetContents :: Handle -> IO String
hPutStrLn :: Handle -> String -> IO () 
\end{verbatim}
\subsection*{Mônadas}
\label{sec:orgb0964bc}

\begin{itemize}
\item Utilizamos as funções anteriores para leitura / escrita de dados usando I/O.

\item Composição feita utilização notação \textbf{do}
\end{itemize}
\subsection*{Mônadas}
\label{sec:org0013c91}

\begin{itemize}
\item Exemplo: Confirmar a reescrita de arquivos de saída.
\end{itemize}

\begin{verbatim}
confirmOverwrite :: FilePath -> IO Bool
confirmOverwrite path
  = do
      putStrLn $ "File:\n" ++ path ++ "\nexists. Confirm overwrite?"
      answer <- getLine
      case answer of
        "y" -> return True
        "n" -> return False
        _   -> do
                 putStrLn "Invalid answer! Please type y or n."
                 confirmOverwrite path
\end{verbatim}
\subsection*{Mônadas}
\label{sec:orgafab664}

\begin{itemize}
\item Com isso a implementação está concluída?
\begin{itemize}
\item Não! Falta a \emph{interação} com o usuário.
\end{itemize}

\item Devemos receber argumentos de linha de comando e processá-los de maneira adequada.
\end{itemize}
\section*{Interação em console}
\label{sec:org2f5c801}

\subsection*{Interação em console}
\label{sec:org10937a4}

\begin{itemize}
\item Qual o formato da entrada esperado pela ferramenta?
\begin{itemize}
\item Devemos especificar o arquivo de entrada e o de saída
\item Se não for especificado, a saída deverá ser o console padrão.
\end{itemize}
\end{itemize}
\subsection*{Interação em console}
\label{sec:orgcb99f2a}

\begin{description}
\item[{Em princípio, podemos utilizar a função \textasciitilde{}getArgs}] IO [String]\textasciitilde{}.
\begin{itemize}
\item Analisamos a lista de \texttt{String} obtida.
\item Validamos se os argumentos estão corretos.
\item Emitir mensagens de erro / ajuda
\end{itemize}
\end{description}
\subsection*{Interação em console}
\label{sec:orgc2c6ab6}

\begin{itemize}
\item Muitas tarefas\ldots{}

\item Melhor utilizar uma biblioteca especializada para isso.
\begin{itemize}
\item Utilizaremos a biblioteca \texttt{optparse-applicative}.
\end{itemize}
\end{itemize}
\subsection*{Interação em console}
\label{sec:org7ff2bb8}

\begin{itemize}
\item Para usar essa biblioteca, devemos:
\begin{itemize}
\item Definir tipos de dados para opções de entrada do programa
\item Definir o parser destas opções
\item Casamento de padrão sobre as opções
\end{itemize}
\end{itemize}
\subsection*{Interação em console}
\label{sec:orga0683fe}

\begin{itemize}
\item Representando o argumento de entrada
\end{itemize}

\begin{verbatim}
data Input
  = Stdin
  | FileInput FilePath
\end{verbatim}
\subsection*{Interação em console}
\label{sec:org5456343}

\begin{itemize}
\item Representando o argumento de saída
\end{itemize}

\begin{verbatim}
data Output
  = Stdout
  | FileOutput FilePath
  deriving Show
\end{verbatim}
\subsection*{Interação em console}
\label{sec:org83fd625}

\begin{itemize}
\item Agrupando entrada e saída
\end{itemize}

\begin{verbatim}
data Options
  = Single Input Output
  deriving Show
\end{verbatim}
\subsection*{Interação em console}
\label{sec:orgc4fb88e}

\begin{itemize}
\item Criando o parser usando \texttt{optparse-applicative}.
\begin{itemize}
\item Lidando apenas com entrada por arquivos.
\end{itemize}
\end{itemize}

\begin{verbatim}
pInputFile :: Parser Input
pInputFile = FileInput <$> parser
  where
    parser =
      strOption
        ( long "input"
          <> short 'i'
          <> metavar "FILE"
          <> help "Input file"
        )
\end{verbatim}
\subsection*{Interação em console}
\label{sec:orgeaed933}

\begin{itemize}
\item Criando o parser para entradas
\item Função \texttt{optional}
\begin{itemize}
\item Retorna \texttt{Nothing} em caso de erro de parsing.
\item Retorna o resultado no construtor \texttt{Just}.
\end{itemize}
\end{itemize}

\begin{verbatim}
pSingleInput :: Parser Input
pSingleInput =
  fromMaybe Stdin <$> optional pInputFile
\end{verbatim}
\subsection*{Interação em console}
\label{sec:orgd042921}

\begin{itemize}
\item Parser para saída segue o mesmo formato.
\end{itemize}
\subsection*{Interação em console}
\label{sec:orgc79486b}

\begin{itemize}
\item Definição do parser.
\end{itemize}

\begin{verbatim}
pSingle :: Parser Options
pSingle =
  Single <$> pSingleInput <*> pSingleOutput
\end{verbatim}
\subsection*{Interação em console}
\label{sec:org378e48f}

\begin{itemize}
\item Para executar o parser, precisamos adicionar informações para mensagens de ajuda.

\item Para isso, devemos criar um valor de tipo \texttt{ParserInfo}
\end{itemize}

\begin{verbatim}
opts :: ParserInfo Options
opts = info (pSingle <**> helper)
            (fullDesc                            <>
             header "BCC328 - Markdown compiler" <>
             progDesc "Simplified Markdown compiler")
\end{verbatim}
\subsection*{Interação em console}
\label{sec:orgd4bdf2d}

\begin{itemize}
\item Com isso, a obtenção de quais opções foram passadas é dada por:
\end{itemize}

\begin{verbatim}
parse :: IO Options
parse = execParser opts
\end{verbatim}
\subsection*{Interação em console}
\label{sec:org62f97c4}

\begin{itemize}
\item O parser construído automatiza tratamento de erros em opções e mensagens de ajuda.
\end{itemize}
\subsection*{Interação em console}
\label{sec:org6561e77}

\begin{itemize}
\item Valores to tipo \texttt{Option} são usados para criar \texttt{Handles}:
\end{itemize}

\begin{verbatim}
createHandles :: Options -> IO (String, Handle, Handle)
createHandles (Single inp out)
  = do
      (title, inpHandle) <- createInputHandle inp
      outHandle <- createOutputHandle out
      return (title, inpHandle, outHandle)
\end{verbatim}
\subsection*{Interação em console}
\label{sec:orgc5a89f7}

\begin{itemize}
\item Criação de \texttt{Handle} de entrada
\end{itemize}

\begin{verbatim}
createInputHandle :: Input -> IO (String, Handle)
createInputHandle Stdin = return ("", stdin)
createInputHandle (FileInput path)
  = (,) path <$> openFile path ReadMode
\end{verbatim}
\subsection*{Interação em console}
\label{sec:org02b93b2}

\begin{itemize}
\item Criação de \texttt{Handle} de saída é similar.
\end{itemize}
\subsection*{Interação em console}
\label{sec:orgd2b240b}

\begin{itemize}
\item Criamos um parser para ler as entradas do compilador.

\item O que falta?
\begin{itemize}
\item Criarmos um pipeline que irá ``ligar'' a casca de IO com o núcleo do compilador.
\end{itemize}
\end{itemize}
\subsection*{Interação em console}
\label{sec:org6a4f19b}

\begin{itemize}
\item Inicialização do pipeline
\begin{itemize}
\item Criação de handles de entrada e de saída
\end{itemize}
\end{itemize}

\begin{verbatim}
startPipeline :: String -> Handle -> Handle -> IO ()
startPipeline title inpHandle outHandle
  = do
      content <- hGetContents inpHandle
      res <- pipeline (title_ title) content
      hPutStrLn outHandle res
\end{verbatim}
\subsection*{Interação em console}
\label{sec:orga9ccd39}

\begin{itemize}
\item Definição do pipeline
\end{itemize}

\begin{verbatim}
pipeline :: Head -> String -> IO String
pipeline title content
  = case frontEnd content of
      Left err -> putStrLn err >> exitFailure
      Right ast -> return $ render $ backEnd title ast
\end{verbatim}
\subsection*{Interação em console}
\label{sec:org835d39d}

\begin{itemize}
\item Definição da função \texttt{main}
\end{itemize}

\begin{verbatim}
main :: IO ()
main = do
  options <- parse
  (title,inpHandle,outHandle) <- createHandles options
  startPipeline title inpHandle outHandle
  hClose inpHandle
  hClose outHandle
\end{verbatim}
\section*{Concluindo}
\label{sec:orge83d2c4}

\subsection*{Concluindo}
\label{sec:org40f2f85}

\begin{itemize}
\item Apresentamos o projeto de um compilador de um versão simplificada de Markdown para HTML.

\item Compilador segue o padrão: functional core / imperative shell
\end{itemize}
\section*{Exercícios}
\label{sec:org750f4eb}

\subsection*{Exercícios}
\label{sec:org89e41c2}

\begin{itemize}
\item Estenda a implementação do compilador desenvolvido para prover
suporte a produção de slides \LaTeX{} usando beamer.
\end{itemize}
\subsection*{Exercícios}
\label{sec:org9628204}

\begin{itemize}
\item Para isso você deverá:
\begin{itemize}
\item Criar uma EDSL para slides beamer.
\item Traduzir documentos markdown para a EDSL
\begin{itemize}
\item Considere que cada header de level 1 é um novo slide.
\end{itemize}
\item Modificar a ``casca'' de IO para incluir a opção de produção de slides.
\end{itemize}
\end{itemize}
\subsection*{Exercícios}
\label{sec:org6596f66}

\begin{itemize}
\item Estender a estrutura de testes para validar a sua implementação.
\end{itemize}
\subsection*{Exercícios}
\label{sec:org80ab502}

\begin{itemize}
\item Você pode considerar útil a documentação da biblioteca \href{https://github.com/pcapriotti/optparse-applicative}{optparse-applicative}.
\end{itemize}
\end{document}
