% Intended LaTeX compiler: pdflatex
\documentclass[11pt]{article}
\usepackage[utf8]{inputenc}
\usepackage[T1]{fontenc}
\usepackage{graphicx}
\usepackage{longtable}
\usepackage{wrapfig}
\usepackage{rotating}
\usepackage[normalem]{ulem}
\usepackage{amsmath}
\usepackage{amssymb}
\usepackage{capt-of}
\usepackage{hyperref}
\author{Construção de compiladores I}
\date{}
\title{Monad Transformers}
\hypersetup{
 pdfauthor={Construção de compiladores I},
 pdftitle={Monad Transformers},
 pdfkeywords={},
 pdfsubject={},
 pdfcreator={Emacs 28.2 (Org mode 9.7)}, 
 pdflang={English}}
\begin{document}

\maketitle
\section*{Objetivos}
\label{sec:org1bfeb9d}

\subsection*{Objetivos}
\label{sec:orgfca5cda}

\begin{itemize}
\item Apresentar o conceito de Monad Transformer para combinar diferentes
efeitos colaterais em um programa.

\item Uso de monad transformers para permitir a manipulação de diretórios no
compilador de markdown.
\end{itemize}
\section*{Markdown}
\label{sec:orgb96e407}

\subsection*{Markdown}
\label{sec:org6270bc1}

\begin{itemize}
\item Iniciamos nosso curso com o projeto de um mini-compilador de Markdown.

\item Objetivo: revisar conceitos da linguagem Haskell
\end{itemize}
\subsection*{Markdown}
\label{sec:org6efbc94}

\begin{itemize}
\item Na última aula, finalizamos uma versão que processa apenas um único arquivo.

\item Nesta aula estenderemos esse compilador com:
\begin{itemize}
\item Processamento de diretórios
\item Incluir folhas de estilo nos html gerados.
\end{itemize}
\end{itemize}
\subsection*{Markdown}
\label{sec:orgd680866}

\begin{itemize}
\item Estender o compilador para processar diretórios é uma tarefa bem direta.
\end{itemize}
\subsection*{Markdown}
\label{sec:org4e3ab71}

\begin{itemize}
\item Incluir folhas de estilo
\begin{itemize}
\item Tarefa simples, mas que leva a código longe do ideal.
\item Mostraremos como melhorar o código combinando mônadas.
\end{itemize}
\end{itemize}
\section*{Processando diretórios}
\label{sec:org31b1d4d}

\subsection*{Processando diretórios}
\label{sec:org3244b0f}

\begin{itemize}
\item Para permitir processar todos os arquivos em um diretório, devemos modificar a interação por consoles.
\end{itemize}
\subsection*{Processando diretórios}
\label{sec:orgd478255}

\begin{itemize}
\item Modificação do tipo de opções de entrada.
\end{itemize}

\begin{verbatim}
data Options
  = Single Input Output
  | Directory FilePath FilePath
  deriving Show
\end{verbatim}
\subsection*{Processando diretórios}
\label{sec:org5261a58}

\begin{itemize}
\item Modificação do parser
\end{itemize}

\begin{verbatim}
pOptions :: Parser Options
pOptions
  = subparser (singleParser <> directoryParser)
    where
      singleParser
        = command "file"
                  (info (helper <*> pSingle)
                        (progDesc "Convert a single markdown file to HTML"))
      directoryParser
        = command "directory"
                  (info (helper <*> pDirectory)
                        (progDesc "Convert all directory md files to HTML"))
\end{verbatim}
\subsection*{Processando diretórios}
\label{sec:org96790e4}

\begin{itemize}
\item Parser de diretórios
\end{itemize}

\begin{verbatim}
pDirectory :: Parser Options
pDirectory =
  Directory <$> pInputDir <*> pOutputDir
\end{verbatim}
\subsection*{Processando diretórios}
\label{sec:org1e0ca8f}

\begin{itemize}
\item Parser de diretório de entrada
\end{itemize}

\begin{verbatim}
pInputDir :: Parser FilePath
pInputDir =
  strOption
    ( long "input"
      <> short 'i'
      <> metavar "DIRECTORY"
      <> help "Input directory"
    )
\end{verbatim}
\subsection*{Processando diretórios}
\label{sec:org0973ba8}

\begin{itemize}
\item Processamento de diretórios de saída é similar.
\end{itemize}
\subsection*{Processando diretórios}
\label{sec:org894ad54}

\begin{itemize}
\item Mudanças no pipeline de compilação.
\end{itemize}

\begin{verbatim}
startPipeline :: IO ()
startPipeline
  = do
      options <- optionsParser
      case options of
        Single inp out ->
          filePipeline False inp out
        Directory inpDir outDir ->
          directoryPipeline inpDir outDir
\end{verbatim}
\subsection*{Processando diretórios}
\label{sec:orge85774f}

\begin{itemize}
\item Modificações no pipeline de arquivos individuais
\end{itemize}

\begin{verbatim}
filePipeline :: Bool -> Input -> Output -> IO ()
filePipeline dirMode inpFile outFile 
  = do
      progressMessage dirMode inpFile
      (title,inpHandle,outHandle) <- createHandles inpFile outFile
      content <- hGetContents inpHandle
      res <- pipeline (title_ title) content
      hPutStrLn outHandle res
      hClose inpHandle
      hClose outHandle
\end{verbatim}
\subsection*{Processando diretórios}
\label{sec:org9256d4b}

\begin{itemize}
\item Pipeline de diretórios
\end{itemize}

\begin{verbatim}
directoryPipeline :: FilePath -> FilePath -> IO ()
directoryPipeline inputDir outputDir
  = do
      d <- directoryContents inputDir
      let files = filesToCompile d
          otherFiles = filesToCopy d
      let entries = map (\ (i,o) -> (FileInput i, FileOutput o)) files
      shouldContinue <- createOutputDirectory outputDir
      unless shouldContinue (hPutStrLn stderr "Cancelled." *> exitFailure)
      mapM_ (uncurry (filePipeline True)) entries
      let copy file = copyFile file (outputDir </> takeFileName file)
      mapM_ copy otherFiles
      putStrLn "Done."
\end{verbatim}
\subsection*{Processando diretórios}
\label{sec:orge1a0d90}

\begin{itemize}
\item Código contendo exatamente essas modificações pode ser encontrado na branch \texttt{markup-directories}.
\end{itemize}
\section*{Adicionando estilos}
\label{sec:org2394107}

\subsection*{Adicionando estilos}
\label{sec:org76588a5}

\begin{itemize}
\item Até o presente momento, não usamos folhas de estilo para formatar o HTML gerado.

\item Como adicionar folhas de estilo?
\end{itemize}
\subsection*{Adicionando estilos}
\label{sec:org470878c}

\begin{itemize}
\item Definido um tipo para armazenar o caminho de folhas de estilo
\end{itemize}

\begin{verbatim}
data Env
  = Env {
      stylePath :: FilePath
    } deriving Show

defaultEnv :: Env
defaultEnv = Env ""
\end{verbatim}
\subsection*{Adicionando estilos}
\label{sec:orgfe3e968}

\begin{itemize}
\item Agora, temos que modificar todo o código para permitir este parâmetro adicional.
\end{itemize}
\subsection*{Adicionando estilos}
\label{sec:org4eb7be8}

\begin{itemize}
\item Adicionando tags para estilos na EDSL de HTML
\end{itemize}

\begin{verbatim}
stylesheet_ :: FilePath -> Head
stylesheet_ path =
    Head $ "<link rel=\"stylesheet\" type=\"text/css\" href=\"" <>
            escape path <> "\">"
\end{verbatim}
\subsection*{Adicionando estilos}
\label{sec:orgf81373c}

\begin{itemize}
\item Modificando o parser de diretórios para entrada
\end{itemize}

\begin{verbatim}
data Options
  = Single Input Output
  | Directory FilePath FilePath Env 
  deriving Show
\end{verbatim}
\subsection*{Adicionando estilos}
\label{sec:org230c5b9}

\begin{itemize}
\item Modificando o parser de diretórios para entrada
\end{itemize}

\begin{verbatim}
pDirectory :: Parser Options
pDirectory =
  Directory <$> pInputDir <*> pOutputDir <*> pEnv
\end{verbatim}
\subsection*{Adicionando estilos}
\label{sec:org99c329b}

\begin{itemize}
\item Modificando o parser de diretórios para entrada
\end{itemize}

\begin{verbatim}
pEnv :: Parser Env
pEnv = fromMaybe defaultEnv <$> optional p
  where
    p =  Env <$> strOption
                 (  long "style"
                 <> short 'S'
                 <> metavar "FILE"
                 <> help "Stylesheet filename"
                 )
\end{verbatim}
\subsection*{Adicionando estilos}
\label{sec:org7aa5503}

\begin{itemize}
\item Modificando o pipeline de compilação
\begin{itemize}
\item Inclusão do environment em \textbf{TODAS} as funções do pipeline.
\end{itemize}
\end{itemize}

\begin{verbatim}
startPipeline :: IO ()
startPipeline
  = do
      options <- optionsParser
      case options of
        Single inp out ->
          filePipeline False defaultEnv inp out
        Directory inpDir outDir env ->
          directoryPipeline env inpDir outDir
\end{verbatim}
\subsection*{Adicionando estilos}
\label{sec:org12bad17}

\begin{itemize}
\item Modificando o pipeline de compilação.
\begin{itemize}
\item Estilos usados apenas em \texttt{filePipeline}.
\item Demais funções apenas ``passam'' o valor de \texttt{Env}.
\end{itemize}
\end{itemize}

\begin{verbatim}
filePipeline :: Bool -> Env -> Input -> Output -> IO ()
filePipeline dirMode env inpFile outFile 
  = do
      progressMessage dirMode inpFile
      (title,inpHandle,outHandle) <- createHandles inpFile outFile
      content <- hGetContents inpHandle
      let header = title_ title <> stylesheet_ (stylePath env) 
      res <- pipeline header content
      hPutStrLn outHandle res
      hClose inpHandle
      hClose outHandle
\end{verbatim}
\subsection*{Adicionando estilos}
\label{sec:orge904aa9}

\begin{itemize}
\item Versão contendo o código com passagem explícita de \texttt{Env} está disponível
na branch \texttt{markup-explicit-env}.
\end{itemize}
\subsection*{Adicionando estilos}
\label{sec:orgc34c931}

\begin{itemize}
\item Essa passagem de valores é tediosa e propensa a erros.

\item Ideal: Acessar o valor somente no ponto onde este será utilizado.
\begin{itemize}
\item Garantir que esse valor não será modificado.
\end{itemize}
\end{itemize}
\subsection*{Adicionando estilos}
\label{sec:orgba08e83}

\begin{itemize}
\item Podemos garantir a situação ideal utilizando a mônada de somente leitura
\begin{itemize}
\item Reader
\end{itemize}
\item Como combinar essa mônada com a mônada de IO?
\end{itemize}
\subsection*{Adicionando estilos}
\label{sec:orgd8ba349}

\begin{itemize}
\item Para isso, devemos utilizar \textbf{transformadores monádicos}.
\begin{itemize}
\item Permitem combinar a funcionalidade de diferentes mônadas.
\end{itemize}
\end{itemize}
\subsection*{Adicionando estilos}
\label{sec:org188357f}

\begin{itemize}
\item Principal função da mônada de somente leitura
\begin{itemize}
\item Retorna o valor armazenado na mônada para consulta.
\end{itemize}
\end{itemize}

\begin{verbatim}
ask :: MonadReader m => m a
\end{verbatim}
\subsection*{Adicionando estilos}
\label{sec:org869b588}

\begin{itemize}
\item Modificando o pipeline de compilação para usar a nova mônada
\end{itemize}

\begin{verbatim}
type CompilerM a = (ReaderT Env IO) a

runCompilerM :: Env -> CompilerM a -> IO a
runCompilerM env m = runReaderT m env
\end{verbatim}
\subsection*{Adicionando estilos}
\label{sec:org15290e9}

\begin{itemize}
\item Modificando o início do pipeline
\end{itemize}

\begin{verbatim}
startPipeline :: IO ()
startPipeline
  = do
      options <- optionsParser
      case options of
        Single inp out ->
          runCompilerM defaultEnv (filePipeline False inp out)
        Directory inpDir outDir env ->
          runCompilerM env (directoryPipeline inpDir outDir)
\end{verbatim}
\subsection*{Adicionando estilos}
\label{sec:org190af6e}

\begin{itemize}
\item Modificando o pipeline de arquivos individuais
\begin{itemize}
\item Uso da função \texttt{ask}
\end{itemize}
\end{itemize}

\begin{verbatim}
filePipeline :: Bool -> Input -> Output -> CompilerM ()
filePipeline dirMode inpFile outFile 
  = do
      env <- ask
      progressMessage dirMode inpFile
      (title,inpHandle,outHandle) <- liftIO $ createHandles inpFile outFile
      content <- liftIO $ hGetContents inpHandle
      let header = title_ title <> stylesheet_ (stylePath env) 
      res <- pipeline header content
      writeAndCloseHandles res inpHandle outHandle
\end{verbatim}
\subsection*{Adicionando estilos}
\label{sec:orgeba9b3d}

\begin{itemize}
\item Demais funções tiveram alterações pontuais
\begin{itemize}
\item Modificação da assinatura de tipos
\item Uso de \texttt{liftIO}.
\end{itemize}
\end{itemize}
\subsection*{Adicionando estilos}
\label{sec:orgc3ff6f3}

\begin{itemize}
\item Código contendo essa versão está disponível na branch \texttt{main}.
\end{itemize}
\section*{Concluindo}
\label{sec:orgf6d7850}

\subsection*{Concluindo}
\label{sec:orgb27dd83}

\begin{itemize}
\item Com isso, terminamos nossa revisão de Haskell.
\item Produzimos um compilador de um subconjunto de Markdown para HTML.
\begin{itemize}
\item Utilizamos várias bibliotecas úteis de Haskell.
\item Apresentamos o padrão functional core / imperative shell
\end{itemize}
\end{itemize}
\subsection*{Concluindo}
\label{sec:org07b93cb}

\begin{itemize}
\item Utilizamos monad transformers em um exemplo muito simples.

\item Recomendo ver um exemplo um pouco mais interessante disponível no seguinte \href{https://github.com/mgrabmueller/TransformersStepByStep}{repositório}.
\end{itemize}
\subsection*{Concluindo}
\label{sec:org9a15193}

\begin{itemize}
\item Próxima aula: iniciamos o primeiro tópico de compiladores
\begin{itemize}
\item Análise léxica
\end{itemize}
\end{itemize}
\section*{Exercícios}
\label{sec:org7504a5b}

\subsection*{Exercícios}
\label{sec:org8963340}

\begin{itemize}
\item Em exercícios anteriores, você extendeu o compilador de Markdown para gerar slides beamer.
\begin{itemize}
\item Modifique a versão atual para gerar slides para todos os arquivos em um diretório
\end{itemize}
\end{itemize}
\subsection*{Exercícios}
\label{sec:org960fd16}

\begin{itemize}
\item Modifique o parser de argumentos de linha de comando para permitir a especificação do
template de slides a ser usado como argumento de linha de comando.
\begin{itemize}
\item \href{https://latex-beamer.com/tutorials/beamer-themes/}{Lista de templates disponíveis}.
\end{itemize}
\item Entrega via Github classroom.
\end{itemize}
\end{document}
