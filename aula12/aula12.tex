% Intended LaTeX compiler: pdflatex
\documentclass[11pt]{article}
\usepackage[utf8]{inputenc}
\usepackage[T1]{fontenc}
\usepackage{graphicx}
\usepackage{longtable}
\usepackage{wrapfig}
\usepackage{rotating}
\usepackage[normalem]{ulem}
\usepackage{amsmath}
\usepackage{amssymb}
\usepackage{capt-of}
\usepackage{hyperref}
\author{Construção de compiladores I}
\date{}
\title{Análise Ascendente}
\hypersetup{
 pdfauthor={Construção de compiladores I},
 pdftitle={Análise Ascendente},
 pdfkeywords={},
 pdfsubject={},
 pdfcreator={Emacs 28.2 (Org mode 9.7)}, 
 pdflang={English}}
\begin{document}

\maketitle
\section*{Objetivos}
\label{sec:org846d29b}

\subsection*{Objetivos}
\label{sec:orgeae88ac}

\begin{itemize}
\item Apresentar uma motivação para o uso de analisadores ascendentes.

\item Introduzir os as funções de fechamento e goto em analisadores LR(0).
\end{itemize}
\section*{Introdução}
\label{sec:orgad79db8}

\subsection*{Introdução}
\label{sec:orgb901c23}

\begin{itemize}
\item Até o presente, vimos técnicas de análise sintática top-down.

\item Intuitivamente, essas técnicas produzem uma derivação iniciando no
símbolo de partida da gramática.
\end{itemize}
\subsection*{Introdução}
\label{sec:org34cf22e}

\begin{itemize}
\item Analisadores descendentes recursivos
\begin{itemize}
\item Simples implementação manual.
\item Restrições sobre a forma da gramática.
\end{itemize}
\end{itemize}
\subsection*{Introdução}
\label{sec:orgea1a920}

\begin{itemize}
\item Parsing expression grammars
\begin{itemize}
\item Não suportam recursão à esquerda.
\item Semântica não intuitiva.
\end{itemize}
\end{itemize}
\subsection*{Introdução}
\label{sec:orgcbe5ca0}

\begin{itemize}
\item Analisador LL(1)
\begin{itemize}
\item Constrói uma derivação mais a esquerda.
\item Utiliza uma tabela construída utilizando os conjuntos first e follow.
\end{itemize}
\end{itemize}
\subsection*{Introdução}
\label{sec:orge56f22d}

\begin{itemize}
\item Problemas com gramáticas LL(1):
\begin{itemize}
\item Parser deve tomar decisão sobre qual produção usar com base nos
primeiro token da entrada.
\end{itemize}
\end{itemize}
\subsection*{Introdução}
\label{sec:org922bc31}

\begin{itemize}
\item Mas, existe outra alternativa?
\end{itemize}
\subsection*{Introdução}
\label{sec:org05d6357}

\begin{itemize}
\item Outra possibilidade: Análise ascendente.
\begin{itemize}
\item Construir uma derivação mais a direita ao \textbf{inverso}.
\end{itemize}
\end{itemize}
\subsection*{Introdução}
\label{sec:org4fc9124}

\begin{itemize}
\item Derivação mais à direita
\begin{itemize}
\item Expandir o não terminal mais a direita em cada passo.
\end{itemize}
\end{itemize}
\subsection*{Introdução}
\label{sec:orgc4ec4c8}

\begin{itemize}
\item Gramática de exemplo
\end{itemize}

\begin{array}{lcl}
  E & \to & E \textbf{+} E \,|\,E \textbf{*} E\,|\,id 
\end{array}
\subsection*{Introdução}
\label{sec:org4abe8d3}

\begin{itemize}
\item Derivação mais a direita para id * id:
\end{itemize}

\begin{array}{lc}
E      & \Rightarrow^R \\
\end{array}
\subsection*{Introdução}
\label{sec:org93f276e}

\begin{itemize}
\item Derivação mais a direita para id * id:
\end{itemize}

\begin{array}{lc}
E      & \Rightarrow^R \\
E * E  & \Rightarrow^R \\
\end{array}
\subsection*{Introdução}
\label{sec:org66dc30c}

\begin{itemize}
\item Derivação mais a direita para id * id:
\end{itemize}

\begin{array}{lc}
E      & \Rightarrow^R \\
E * E  & \Rightarrow^R \\
E * id & \Rightarrow^R \\
\end{array}
\subsection*{Introdução}
\label{sec:orgce11548}

\begin{itemize}
\item Derivação mais a direita para id * id:
\end{itemize}

\begin{array}{lc}
E      & \Rightarrow^R \\
E * E  & \Rightarrow^R \\
E * id & \Rightarrow^R \\
id * id \\
\end{array}
\subsection*{Introdução}
\label{sec:org590befd}

\begin{itemize}
\item Mas, como um analisador ascendente functiona?
\end{itemize}
\subsection*{Introdução}
\label{sec:org03c7b9c}

\begin{itemize}
\item O analisador utiliza:
\begin{itemize}
\item Sequência de tokens de entrada.
\item Pilha.
\end{itemize}
\end{itemize}
\subsection*{Introdução}
\label{sec:org740311a}

\begin{itemize}
\item O analisador pode utilizar 4 tipos de ação:
\begin{itemize}
\item shift: mover itens da entrada para a pilha
\item reduce: escolher regra \(A \to \beta\)
\begin{itemize}
\item remover \(\beta\) do topo da pilha
\item empilhar \(A\).
\end{itemize}
\item aceitar e falhar.
\end{itemize}
\end{itemize}
\subsection*{Introdução}
\label{sec:org0d95eb5}

\begin{itemize}
\item Idealmente, o analisador deve reduzir uma palavra de entrada até o símbolo de partida da gramática.

\item Vamos mostrar o funcionamento, usando um exemplo.
\end{itemize}
\subsection*{Introdução}
\label{sec:orgc101255}

\begin{itemize}
\item Exemplo:
\begin{itemize}
\item entrada: id * id\$
\item pilha: \$
\end{itemize}
\end{itemize}
\subsection*{Introdução}
\label{sec:org2975fdf}

\begin{itemize}
\item Ação: shift id para a pilha.

\item Exemplo:
\begin{itemize}
\item entrada: * id\$
\item pilha: \$ id
\end{itemize}
\end{itemize}
\subsection*{Introdução}
\label{sec:orgbdba4d5}

\begin{itemize}
\item Ação: reduzir id usando \(E \to id\).

\item Exemplo:
\begin{itemize}
\item entrada: * id\$
\item pilha: \$ E
\end{itemize}
\end{itemize}
\subsection*{Introdução}
\label{sec:org5e36856}

\begin{itemize}
\item Ação: shift * para a pilha.

\item Exemplo:
\begin{itemize}
\item entrada: id\$
\item pilha: \$ E *
\end{itemize}
\end{itemize}
\subsection*{Introdução}
\label{sec:org21469ad}

\begin{itemize}
\item Ação: shift id para a pilha.

\item Exemplo:
\begin{itemize}
\item entrada: \$
\item pilha: \$ E * id
\end{itemize}
\end{itemize}
\subsection*{Introdução}
\label{sec:org74d4882}

\begin{itemize}
\item Ação: reduce id usando \(E\to id\).

\item Exemplo:
\begin{itemize}
\item entrada: \$
\item pilha: \$ E * E
\end{itemize}
\end{itemize}
\subsection*{Introdução}
\label{sec:org81807ca}

\begin{itemize}
\item Ação: reduce E * E usando \(E\to E * E\).

\item Exemplo:
\begin{itemize}
\item entrada: \$
\item pilha: \$ E
\end{itemize}
\end{itemize}
\subsection*{Introdução}
\label{sec:orgfd27892}

\begin{itemize}
\item Tendo a entrada sido consumida e a pilha é formada apenas pelo símbolo de partida e o marcador de final de pilha,
temos que a palavra é aceita.
\end{itemize}
\subsection*{Introdução}
\label{sec:org077efeb}

\begin{itemize}
\item Observe que o analisador produz uma derivação mais a direita \textbf{invertida}
\end{itemize}
\subsection*{Introdução}
\label{sec:org425ea64}

\begin{itemize}
\item Porém, como determinar qual ação deve ser executada?
\end{itemize}
\subsection*{Introdução}
\label{sec:org8695d98}

\begin{itemize}
\item Analisadores ascendentes usam um AFD sobre a pilha para
determinar qual ação executar.

\item AFD representado por uma tabela que armazena quais ações
devem ser executadas pelo analisador.
\end{itemize}
\subsection*{Introdução}
\label{sec:org1f138e5}

\begin{itemize}
\item Diferentes técnicas de análise ascendente usam diferentes
estratégias para construção das tabelas.

\item Nesta aula, veremos o analisador LR(0)
\end{itemize}
\section*{Analisador LR(0)}
\label{sec:orgd6e76bf}

\subsection*{Analisador LR(0)}
\label{sec:orgb3df8fa}

\begin{itemize}
\item Analisador sintático ascendente que usa apenas a pilha para decidir ações.

\item Não aplicável em gramáticas de linguagens de programação.
\begin{itemize}
\item Útil para compreensão do mecanismo de construção de tabelas.
\end{itemize}
\end{itemize}
\subsection*{Analisador LR(0)}
\label{sec:orgd6f6b3e}

\begin{itemize}
\item A construção de tabelas LR(0) utiliza o conceito de \textbf{item}
\begin{itemize}
\item Item: regra de uma gramática contendo uma marcação em seu lado direito.
\item Marcação representada por um ``.''
\end{itemize}
\end{itemize}
\subsection*{Analisador LR(0)}
\label{sec:orga998d9e}

\begin{itemize}
\item A ideia do algoritmo é construir um AFD sobre coleções de itens.

\item Cada estado do AFD representa um conjunto de itens.
\end{itemize}
\subsection*{Analisador LR(0)}
\label{sec:org42e4bc5}

\begin{itemize}
\item Como obter o conjunto de itens?

\item Primeiro, precisamos modificar a gramática de entrada e calculamos o fechamento e
transição entre conjuntos de itens.
\end{itemize}
\subsection*{Analisador LR(0)}
\label{sec:org30dc8ec}

\begin{itemize}
\item Fechamento do conjunto de itens \(I\).
\begin{itemize}
\item \(I\subseteq closure(I)\).
\item Se \(A \to \alpha \textbf{.}B\beta \in I\), incluir toda regra \(B \to \gamma\) em I.
\item Repetir passo anterior enquanto possível.
\end{itemize}
\end{itemize}
\subsection*{Analisador LR(0)}
\label{sec:orgaa463cc}

\begin{itemize}
\item Outro ponto da construção do autômato LR(0) é o cálculo da função de transição.

\item \(goto(I,a)\)
\begin{itemize}
\item \(I\): conjunto de itens.
\item \(a \in \Sigma\).
\end{itemize}
\end{itemize}
\subsection*{Analisador LR(0)}
\label{sec:org82b1b54}

\begin{itemize}
\item Definimos \(goto(I,X)\):
\begin{itemize}
\item \(J \leftarrow \emptyset\)
\item Para cada item \(A \to \alpha .X\beta \in I\)
\begin{itemize}
\item Adicione \(A \to \alpha X.\beta\) a \(J\).
\end{itemize}
\item retorne \(closure(J)\).
\end{itemize}
\end{itemize}
\subsection*{Analisador LR(0)}
\label{sec:orgf9ccaf6}

\begin{itemize}
\item Construção da tabela
\begin{itemize}
\item Inicialize \(T\) com \{closure(\{S' \(\to\) .S \$\})\}.
\item Inicialize \(E\) com \(\emptyset\)
\end{itemize}
\end{itemize}
\subsection*{Analisador LR(0)}
\label{sec:org9d02f4e}

\begin{itemize}
\item Repita enquanto T e E mudarem
\begin{itemize}
\item para cada I \(\in\) T
\begin{itemize}
\item para cada A \(\to\) \(\alpha\) .X \(\beta\) \(\in\) I
\begin{itemize}
\item J = goto(I,X)
\item T = T \(\cup\) \{J\}
\item E = E \(\cup\) \{(I, X , J)\}
\end{itemize}
\end{itemize}
\end{itemize}
\end{itemize}
\subsection*{Analisador LR(0)}
\label{sec:orga5db731}

\begin{itemize}
\item Ilustraremos essas etapas considerando a seguinte gramática.
\end{itemize}

\begin{array}{lcl}
  S & \to & \textbf{(}L\textbf{)}\,|\, \textbf{x}\\
  L & \to & L\,\textbf{,}\,S\,|\,S\\
\end{array}
\subsection*{Analisador LR(0)}
\label{sec:orgf13f75a}

\begin{itemize}
\item Algoritmo de análise

\item Inicialize a entrada com w\$
\item Inicialize a pilha com \$0
\end{itemize}
\subsection*{Analisador LR(0)}
\label{sec:org300f9f3}

\begin{itemize}
\item repita enquanto possível
\begin{itemize}
\item Seja \emph{a} o 1o token da entrada
\item Seja \emph{t} o topo da pilha.
\item Se A[t,a] = shift n
\begin{itemize}
\item Empilhe n
\item \emph{a} passa a ser o próximo token
\end{itemize}
\end{itemize}
\end{itemize}
\subsection*{Analisador LR(0)}
\label{sec:org1fe43a0}

\begin{itemize}
\item se A[t,a] = reduce A \(\to\) \(\gamma\)
\begin{itemize}
\item Desempilhe \(|\gamma|\) itens.
\item Seja \emph{p} o topo da pilha
\item empilhe G[t, A]
\end{itemize}
\end{itemize}
\subsection*{Analisador LR(0)}
\label{sec:org9fe6def}

\begin{itemize}
\item Se A[t, a] = accept, aceite senão erro.
\end{itemize}
\section*{Exercícios}
\label{sec:orgc9b7d52}

\subsection*{Exercícios}
\label{sec:org23449fa}

\begin{itemize}
\item Apresente o conjunto de itens obtidos para a versão estendida da seguinte gramática:
\begin{itemize}
\item Considere como conjunto inicial o formado pela nova variável de partida.
\end{itemize}
\end{itemize}

\begin{array}{lcl}
E & \to & T \textbf{+} E\,|\,T\\
T & \to & \textbf{x}\\
\end{array}
\end{document}
