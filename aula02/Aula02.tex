% Intended LaTeX compiler: pdflatex
\documentclass[11pt]{article}
\usepackage[utf8]{inputenc}
\usepackage[T1]{fontenc}
\usepackage{graphicx}
\usepackage{longtable}
\usepackage{wrapfig}
\usepackage{rotating}
\usepackage[normalem]{ulem}
\usepackage{amsmath}
\usepackage{amssymb}
\usepackage{capt-of}
\usepackage{hyperref}
\author{Construção de compiladores I}
\date{}
\title{Revisão de Haskell}
\hypersetup{
 pdfauthor={Construção de compiladores I},
 pdftitle={Revisão de Haskell},
 pdfkeywords={},
 pdfsubject={},
 pdfcreator={Emacs 28.2 (Org mode 9.7)}, 
 pdflang={English}}
\begin{document}

\maketitle
\section*{Objetivos}
\label{sec:org4edcd26}

\subsection*{Objetivos}
\label{sec:org524006c}

\begin{itemize}
\item Revisar conceitos sobre functores aplicativos.

\item Construção de parsers usando functores aplicativos.
\end{itemize}
\section*{Markdown}
\label{sec:org0c7e72c}

\subsection*{Markdown}
\label{sec:orgee23d2f}

\begin{itemize}
\item Na última aula, iniciamos a construção de um compilador de Markdown para HTML
\begin{itemize}
\item Criação de uma EDSL para Html
\item Uso da EDSL para produzir Html a partir da AST de Markdown
\end{itemize}
\end{itemize}
\subsection*{Markdown}
\label{sec:org3712707}

\begin{itemize}
\item Porém, como produzir a AST a partir de texto armazenado em arquivos?

\item Como manipular argumentos de entrada para o programa?
\end{itemize}
\subsection*{Markdown}
\label{sec:org245f9a2}

\begin{itemize}
\item Para produzir a AST, precisamos construir um \textbf{parser}.

\item Parser: função de tipo \texttt{String -> AST}.
\end{itemize}
\subsection*{Markdown}
\label{sec:org4c4eb12}

\begin{itemize}
\item Idealmente, para especificarmos um parser, precisamos de uma gramática.
\end{itemize}
\section*{Gramáticas}
\label{sec:org040ed4c}

\subsection*{Gramáticas}
\label{sec:org94bf7a0}

\begin{itemize}
\item Uma gramática livre de contexto \(G=(V,\Sigma,R,P)\):
\begin{itemize}
\item \(V\): conjunto finito de variáveis
\item \(\Sigma\): alfabeto
\item \(R\subseteq V \times (V\cup\Sigma)^*\): regras
\item \(P\in V\): variável de partida.
\end{itemize}
\end{itemize}
\subsection*{Gramáticas}
\label{sec:org74e5ed4}

\begin{itemize}
\item Exemplo: gramática livre de contexto para palavras com a mesma quantidade de 0s e 1s:
\end{itemize}

\begin{array}{lcl}
   P & \to & 0P1P \,|\,1P0P\,|\,\lambda
\end{array}
\subsection*{Gramáticas}
\label{sec:org2a9da0d}

\begin{itemize}
\item Na gramática anterior, temos:
\begin{itemize}
\item \(V=\{P\}\), \(\Sigma=\{0,1\}\)
\item Três regras:
\end{itemize}
\end{itemize}

\begin{array}{l}
  P \to 0P1P\\
  P \to 1P0P\\
  P \to \lambda\\
\end{array}
\subsection*{Gramáticas}
\label{sec:orged49163}

\begin{itemize}
\item Representando a AST:
\end{itemize}

\begin{verbatim}
data AST
  = Zero AST AST -- starts with 0
  | One AST AST  -- starts with 1  
  | Empty        -- empty string
\end{verbatim}
\subsection*{Gramáticas}
\label{sec:orgc2cd555}

\begin{itemize}
\item Como construir um parser para esta gramática?
\end{itemize}

\begin{verbatim}
parse :: String -> AST
\end{verbatim}
\subsection*{Gramáticas}
\label{sec:orgad27e7a}

\begin{itemize}
\item Tentando implementar diretamente:
\end{itemize}

\begin{verbatim}
parse :: String -> AST
parse = fst . parse'
   where
      parse' [] = (Empty, [])
      parse' ('0' : rest) = 
        let (t1,r1) = parse' rest
            (t2,r2) = parse' r1 
        in (Zero t1 t2, r2)
      parse' ('1' : rest) = 
        let (t1,r1) = parse' rest
            (t2,r2) = parse' r1 
        in (One t1 t2, r2)
      parse' _ = ???
\end{verbatim}
\subsection*{Gramáticas}
\label{sec:orgcdb7b79}

\begin{itemize}
\item Problemas da implementação anterior:
\begin{itemize}
\item Passagem explícita do restante da string.
\item Repetição de código.
\item Como lidar com erros?
\end{itemize}
\end{itemize}
\subsection*{Gramáticas}
\label{sec:orgae2dd6c}

\begin{itemize}
\item Lidar com manipulação de erros tende a
aumentar a complexidade do código.
\end{itemize}

\begin{verbatim}
parse' [] = Just (Empty, [])
parse' ('0' : rest) =
   case parse' rest of
     Just (t1,r1) ->
       case parse' r1 of
          Just (t2,r2) -> Just (Zero t1 t2, r2)
          _ -> Nothing 
     _ -> Nothing
\end{verbatim}
\subsection*{Gramáticas}
\label{sec:orgb0c6cff}

\begin{itemize}
\item Para construir parsers, utilizaremos a EDSL da biblioteca \texttt{mega-parsec}.

\item Em essência, utiliza os conceitos de functores aplicativos para
construir parsers.
\end{itemize}
\section*{Functores aplicativos}
\label{sec:orgc13cd4b}

\subsection*{Functores aplicativos}
\label{sec:orgbcc0096}

\begin{itemize}
\item Faremos um pequeno ``desvio'' para revisar o conceito de functor aplicativo.

\item Esse conceito é fundamental para desenvolvimento em Haskell.
\end{itemize}
\subsection*{Functores aplicativos}
\label{sec:org027516d}

\begin{itemize}
\item Functor: Abstração que permite aplicar uma função sobre elementos de uma estrutura.
\end{itemize}

\begin{verbatim}
class Functor f where
  fmap :: (a -> b) -> f a -> f b

(<$>) :: Functor f => (a -> b) -> f a -> f b
(<$>) = fmap
\end{verbatim}
\subsection*{Functores aplicativos}
\label{sec:orgb8a44d5}

\begin{itemize}
\item Exemplo: \texttt{Maybe} é um \texttt{Functor}.
\end{itemize}

\begin{verbatim}
instance Functor Maybe where
  fmap _ Nothing = Nothing
  fmap f (Just v) = Just (f v)
\end{verbatim}
\subsection*{Functores aplicativos}
\label{sec:orgbb10dad}

\begin{itemize}
\item Exemplo:
\end{itemize}

\begin{verbatim}
not <$> Nothing = Nothing
not <$> (Just True) = Just (not True) = Just False
\end{verbatim}
\subsection*{Functores aplicativos}
\label{sec:org036ae7c}

\begin{itemize}
\item Intuitivamente, functores permitem que façamos chamadas de função a
todos os elementos de uma estrutura de dados.
\end{itemize}

\begin{verbatim}
(2 *) <$> Nothing = Nothing
(2 *) <$> (Just 2) = Just (2 * 2) = Just 4
\end{verbatim}
\subsection*{Functores aplicativos}
\label{sec:org0b2afd6}

\begin{itemize}
\item Usamos functores para aplicar funções a elementos \textbf{dentro} de uma estrutura.

\item Pergunta: como realizar chamadas de funções armazenadas dentro de estrutura de dados?
\end{itemize}
\subsection*{Functores aplicativos}
\label{sec:org4fb5c2d}

\begin{itemize}
\item A abstração de functor aplicativo modela a capacidade de chamar funções dentro de uma estrutura de dados
\end{itemize}

\begin{verbatim}
class Functor f => Applicative f where
  pure :: a -> f a
  (<*>) :: f (a -> b) -> f a -> f b
\end{verbatim}
\subsection*{Functores aplicativos}
\label{sec:org64642dd}

\begin{itemize}
\item O principal componente da abstração de functor aplicativo é o operador \texttt{<*>}.
\begin{itemize}
\item Responsável por chamadas de função dentro de estruturas de dados.
\end{itemize}
\end{itemize}

\begin{verbatim}
(<*>) :: f (a -> b) -> f a -> f b
\end{verbatim}
\subsection*{Functores aplicativos}
\label{sec:org3131a0c}

\begin{itemize}
\item O tipo \texttt{Maybe} é um functor aplicativo.
\end{itemize}

\begin{verbatim}
instance Applicative Maybe where  
  pure = Just  
  Nothing  <*> _ = Nothing
  _        <*> Nothing = Nothing
  (Just f) <*> (Just x) = Just (f x)
\end{verbatim}
\subsection*{Functores aplicativos}
\label{sec:orgb211a70}

\begin{itemize}
\item Exemplos
\end{itemize}

\begin{verbatim}
(Just not) <*> Nothing = Nothing
(Just not) <*> (Just False) = Just (not False) = Just True
\end{verbatim}
\subsection*{Functores aplicativos}
\label{sec:org1fde6a0}

\begin{itemize}
\item Exemplos
\end{itemize}

\begin{verbatim}
(Just not) <*> Nothing = Nothing
(Just not) <*> (Just False) = Just (not False) = Just True
\end{verbatim}
\subsection*{Functores aplicativos}
\label{sec:org074a284}

\begin{itemize}
\item Podemos combinar functores e functores aplicativos
\end{itemize}

\begin{verbatim}
(*) <$> (Just 2) <*> (Just 3) =
\end{verbatim}
\subsection*{Functores aplicativos}
\label{sec:orgb95f9e0}

\begin{itemize}
\item Podemos combinar functores e functores aplicativos
\end{itemize}

\begin{verbatim}
(*) <$> (Just 2) <*> (Just 3) =
Just (2 *) <*> (Just 3) =
\end{verbatim}
\subsection*{Functores aplicativos}
\label{sec:org9f0a4ee}

\begin{itemize}
\item Podemos combinar functores e functores aplicativos
\end{itemize}

\begin{verbatim}
(*) <$> (Just 2) <*> (Just 3) =
Just (2 *) <*> (Just 3) =
Just (2 * 3) = Just 6
\end{verbatim}
\subsection*{Functores aplicativos}
\label{sec:orge951718}

\begin{itemize}
\item No contexto de parsing, o operador \texttt{<*>} representa a composição sequencial de parsers.

\item O operador \texttt{<\$>} é utilizado para construir o resultado do parsing.
\end{itemize}
\subsection*{Functores aplicativos}
\label{sec:org3f6caa7}

\begin{itemize}
\item De volta à gramática original:
\end{itemize}

\begin{array}{lcl}
   P & \to & 0P1P \,|\,1P0P\,|\,\lambda
\end{array}
\subsection*{Functores aplicativos}
\label{sec:org5f61d65}

\begin{itemize}
\item E a árvore que representa o resultado do parsing.
\end{itemize}

\begin{verbatim}
data AST
  = Zero AST AST -- starts with 0
  | One AST AST  -- starts with 1  
  | Empty        -- empty string
\end{verbatim}
\subsection*{Functores aplicativos}
\label{sec:org1b6372a}

\begin{itemize}
\item O parsing utilizando a biblioteca mega-parsec é:
\end{itemize}

\begin{verbatim}
type Parser = Parsec Void String

sample :: Parser AST
sample = try startWithZero <|>
         try startWithOne  <|>
         pure Empty

startWithZero :: Parser AST
startWithZero
  = f <$> symbol "0" <*> sample <*> symbol "1"  <*> sample
    where
      f _ t _ t' = Zero t t'

startWithOne :: Parser AST
startWithOne
  = g <$> symbol "1" <*> sample <*> symbol "0"  <*> sample
    where
      g _ t _ t' = One t t'
\end{verbatim}
\subsection*{Functores aplicativos}
\label{sec:orgc2d62c6}

\begin{itemize}
\item Concatenação (sequência) é representada pelo operador \texttt{<*>}.
\item Alternativas são representadas pelo operador \texttt{<|>}.
\end{itemize}
\subsection*{Functores aplicativos}
\label{sec:org686d49e}

\begin{itemize}
\item Símbolos são representados pela função \texttt{symbol}.
\item Função \texttt{try} permite o backtracking.
\begin{itemize}
\item Caso aconteça um erro, a entrada é passada não modificada para próxima alternativa.
\end{itemize}
\item Variáveis da gramática correspondem a chamadas de função.
\end{itemize}
\section*{Parser de Markdown}
\label{sec:orgec28fd6}

\subsection*{Parser de Markdown}
\label{sec:orgfe91d84}

\begin{itemize}
\item Gramática de Markdown
\end{itemize}

\begin{array}{lcl}
Document & \to & Structure^*\\
Structure & \to  & UnorderedList \\
          & \mid & CodeList \\
          & \mid & Heading \\
          & \mid & Paragraph\\
\end{array}
\subsection*{Parser de Markdown}
\label{sec:orgee54625}

\begin{itemize}
\item Gramática de Markdown (continuação)
\end{itemize}

\begin{array}{lcl}
UnorderedList & \to & - \Sigma^*\\
CodeList      & \to & > \Sigma^*\\
Paragraph     & \to & \Sigma^*\\
Heading       & \to & Level\:\:\:Paragraph\\
Level         & \to & * \,|\, **\,|\, ***\\
\end{array}
\subsection*{Parser de Markdown}
\label{sec:org60cb504}

\begin{itemize}
\item De posse da gramática, podemos representar seu parser:
\begin{itemize}
\item Parser \texttt{sc}: remove espaços em branco entre linhas.
\item Linhas são separadas por uma quebra de linha.
\end{itemize}
\end{itemize}

\begin{verbatim}
pDocument :: Parser Document
pDocument = (sc *> pLine) `sepBy` newline
\end{verbatim}
\subsection*{Parser de Markdown}
\label{sec:orgbe96972}

\begin{itemize}
\item Parser de uma linha do arquivo
\end{itemize}

\begin{verbatim}
pLine :: Parser Structure
pLine = choice [ pUnorderedList
               , pCodeBlock
               , pHeading
               , pParagraph
               ]
\end{verbatim}
\subsection*{Parser de Markdown}
\label{sec:org43de2d9}

\begin{itemize}
\item Parser de linha não ordenada
\end{itemize}

\begin{verbatim}
pUnorderedList :: Parser Structure
pUnorderedList = (UnorderedList . wrap) <$> p
  where
    p = symbol "-" *> pString
\end{verbatim}
\subsection*{Parser de Markdown}
\label{sec:org61c41d5}

\begin{itemize}
\item Parser de parágrafos
\end{itemize}

\begin{verbatim}
pParagraph :: Parser Structure
pParagraph = Paragraph <$> pString

pString :: Parser String
pString = many (satisfy (flip notElem "\n*-#>"))
\end{verbatim}
\subsection*{Parser de Markdown}
\label{sec:org2f5df19}

\begin{itemize}
\item Demais casos, seguem estruturas similares.
\end{itemize}
\section*{Concluindo}
\label{sec:org1fb8c55}

\subsection*{Concluindo}
\label{sec:orga25f746}

\begin{itemize}
\item Nesta aula, revisamos o conceito de functor applicativo e
como este é utilizado para construir parsers na biblioteca
\texttt{megaparsec}.
\end{itemize}
\subsection*{Concluindo}
\label{sec:orgc95670f}

\begin{itemize}
\item Na próxima aula veremos outro padrão importante para estruturar programas: Mônadas.
\end{itemize}
\section*{Exercícios}
\label{sec:org4e1c1f4}

\subsection*{Exercícios}
\label{sec:orgde91ca8}

\begin{itemize}
\item Estenda o parser de Markdown com suporte a
\begin{itemize}
\item Listas ordenadas
\item Links
\item Imagens
\end{itemize}
\end{itemize}
\subsection*{Exercícios}
\label{sec:org76d7308}

\begin{itemize}
\item Gramática para listas ordenadas.
\end{itemize}

\begin{array}{lcl}
OrderedList & \to & \# \Sigma^*\\
\end{array}
\subsection*{Exercícios}
\label{sec:org9d6e9a6}

\begin{itemize}
\item Gramática para links
\end{itemize}

\begin{array}{lcl}
Link & \to & [\Sigma^*](\Sigma^*)\\
\end{array}
\subsection*{Exercícios}
\label{sec:orgec048b1}

\begin{itemize}
\item Gramática para imagens
\end{itemize}

\begin{array}{lcl}
Image & \to & ![\Sigma^*](\Sigma^*)\\
\end{array}
\end{document}
