% Intended LaTeX compiler: pdflatex
\documentclass[11pt]{article}
\usepackage[utf8]{inputenc}
\usepackage[T1]{fontenc}
\usepackage{graphicx}
\usepackage{longtable}
\usepackage{wrapfig}
\usepackage{rotating}
\usepackage[normalem]{ulem}
\usepackage{amsmath}
\usepackage{amssymb}
\usepackage{capt-of}
\usepackage{hyperref}
\author{Construção de compiladores I}
\date{}
\title{Análise LALR}
\hypersetup{
 pdfauthor={Construção de compiladores I},
 pdftitle={Análise LALR},
 pdfkeywords={},
 pdfsubject={},
 pdfcreator={Emacs 28.2 (Org mode 9.7)}, 
 pdflang={English}}
\begin{document}

\maketitle
\section*{Objetivos}
\label{sec:orgde88dae}

\subsection*{Objetivos}
\label{sec:orgba51b5e}

\begin{itemize}
\item Apresentar o algoritmo de análise sintática LALR.
\end{itemize}
\section*{Introdução}
\label{sec:org5aa299a}

\subsection*{Introdução}
\label{sec:org9b23995}

\begin{itemize}
\item Nas aulas anteriores, vimos o algorithm LR(1).

\item Este algoritmo consegue realizar o parsing de construções
de linguagens de programação.
\end{itemize}
\subsection*{Introdução}
\label{sec:orgdacb32a}

\begin{itemize}
\item Problema: o algoritmo LR(1) gera um número muito grande de estados.

\item \textbf{Solução}: combinar conjuntos de itens que diferem apenas pelos \textbf{lookaheads}.
\begin{itemize}
\item Isso é o chamado analisador LALR.
\end{itemize}
\end{itemize}
\subsection*{Introdução}
\label{sec:org339196d}

\begin{itemize}
\item Para combinar conjuntos de itens, é útil considerar o conceito de \textbf{núcleo}.

\item O \textbf{núcleo} de um conjunto de itens é um subconjunto de itens utilizado para criar o conjunto.
\end{itemize}
\subsection*{Introdução}
\label{sec:orgaca6bee}

\begin{itemize}
\item Após criar os conjuntos LR(1), combinamos os conjuntos com o mesmo núcleo em um único.
\end{itemize}
\section*{Construção dos itens LALR}
\label{sec:org406f11d}

\subsection*{Construção dos itens LALR}
\label{sec:org2edec9e}

\begin{itemize}
\item Fechamento de conjunto de itens \(I\).
\begin{itemize}
\item \(I\subseteq closure(I)\).
\item Para cada item \([A\to \alpha \textbf{.}B\beta,a]\) em \(I\)
\begin{itemize}
\item Para cada regra \(B \to \gamma\) em \(G'\)
\begin{itemize}
\item Para cada \(b\in first(\beta a)\)
\begin{itemize}
\item Adicione \([B \to .\gamma,b]\) em \(I\)
\end{itemize}
\end{itemize}
\end{itemize}
\end{itemize}
\item Repita enquanto houver alterações em \(I\).
\end{itemize}
\subsection*{Construção dos itens LALR}
\label{sec:org142c089}

\begin{itemize}
\item Função de \(goto(I,X)\)
\begin{itemize}
\item Inicialize \(J\) como \(\emptyset\).
\item Para cada item \([A\to \alpha .X \beta,a]\) em \(I\)
\begin{itemize}
\item Adicione o item \([A \to \alpha X. \beta, a]\) ao conjunto \(J\).
\end{itemize}
\item retorne \(closure(J)\)
\end{itemize}
\end{itemize}
\subsection*{Construção dos itens LALR}
\label{sec:orgbeb706f}

\begin{itemize}
\item Função de construção de itens \(G'\)
\begin{itemize}
\item inicializa \(C\) como closure(\{[S \(\to\) .S, \$]\})
\begin{itemize}
\item Para cada conjunto \(I \in C\)
\begin{itemize}
\item Para cada símbolo \(X\) de \(G'\)
\begin{itemize}
\item se \(goto(I,X) \neq \emptyset \land goto(I,X) \not\in C\)
\begin{itemize}
\item Adicione \(goto(I,X)\) em \(C\)
\end{itemize}
\end{itemize}
\end{itemize}
\end{itemize}
\item repetir enquanto houver alterações em \(C\).
\end{itemize}
\end{itemize}
\section*{Construção da tabela LALR}
\label{sec:org78be849}

\subsection*{Construção da tabela LALR}
\label{sec:org7b4fe57}

\begin{itemize}
\item Se \([A \to \alpha .a\beta,b] \in I_i\) e \(goto(I_i,a) = I_j\),
\begin{itemize}
\item A[i,a] = shift j.
\end{itemize}
\end{itemize}
\subsection*{Construção da tabela LALR}
\label{sec:orge956b2b}

\begin{itemize}
\item Se \([A \to \alpha . , a] \in I_i\) e \(A \neq S'\)
\begin{itemize}
\item A[i,a] = reduce A \(\to\) \(\alpha\)
\end{itemize}
\end{itemize}
\subsection*{Construção da tabela LALR}
\label{sec:org8cc05c5}

\begin{itemize}
\item Se [S' \(\to\) S., \$] \(\in\) I\textsubscript{i}
\begin{itemize}
\item A[i,\$] = accept
\end{itemize}
\end{itemize}
\subsection*{Construção da tabela LALR}
\label{sec:org2e5769d}

\begin{itemize}
\item Seja J = I\textsubscript{1} \(\cup\) \ldots{} \(\cup\) I\textsubscript{n}.
\begin{itemize}
\item Núcleo de cada I\textsubscript{i} é o mesmo.
\end{itemize}
\item Seja K a união de todos os itens de goto(I\textsubscript{1,X}).
\begin{itemize}
\item Fazemos G[J,X] = K
\end{itemize}
\end{itemize}
\section*{Exemplo}
\label{sec:orged07a21}

\subsection*{Exemplo}
\label{sec:org7d21ac5}

\begin{itemize}
\item Construção da tabela LALR para a gramática
\end{itemize}

\begin{array}{lcl}
   S & \to & ( L ) \,|\, x\\
   L & \to & L , S \,|\, S\\
\end{array}
\section*{Construção eficiente da tabela}
\label{sec:org6c26f55}

\subsection*{Construção eficiente da tabela}
\label{sec:org1136db7}

\begin{itemize}
\item O algoritmo LALR melhora o consumo de memória.
\begin{itemize}
\item Combinar o número de itens.
\end{itemize}

\item Porém, ainda precisamos computar o conjunto
completo de itens.
\end{itemize}
\subsection*{Construção eficiente da tabela}
\label{sec:org60bc0ff}

\begin{itemize}
\item Ao invés de construir os itens LR(1), podemos construir apenas
os núcleos de itens LR(0) e calcular os lookaheads.

\item A partir do núcleo dos itens LR(1), calculamos a tabela.
\end{itemize}
\subsection*{Construção eficiente da tabela}
\label{sec:orga20ce9e}

\begin{itemize}
\item Determinando lookaheads para um núcleo K e um símbolo X.

\item Repita os passos seguintes para cada item
\begin{itemize}
\item A \(\to\) \(\alpha\) . \(\beta\) \(\in\) K.
\end{itemize}
\end{itemize}
\subsection*{Construção eficiente da tabela}
\label{sec:orga06ae93}

\begin{itemize}
\item \(J \leftarrow closure(\{[A\to \alpha . \beta, \#]\})\)
\item Se \([B \to \gamma .X \delta, a] \in J \land a \neq \#\)
\begin{itemize}
\item \(a\) é gerado espontaneamente para goto(I,X).
\end{itemize}
\item Se \([B \to \gamma .X \delta, \#] \in J\)
\begin{itemize}
\item Propague lookaheads de \(A\to \alpha . \beta \in I\) para \(B \to \gamma X.\delta\) em goto(I,X).
\end{itemize}
\end{itemize}
\section*{Exemplo}
\label{sec:org9b7be65}

\subsection*{Exemplo}
\label{sec:org7a40bdb}

\begin{itemize}
\item Construção eficiente da tabela LALR para a gramática
\end{itemize}

\begin{array}{lcl}
   S & \to & ( L ) \,|\, x\\
   L & \to & L , S \,|\, S\\
\end{array}
\section*{Concluindo}
\label{sec:org255e590}

\subsection*{Concluindo}
\label{sec:orga2d0f2b}

\begin{itemize}
\item Nesta aula apresentamos dois algoritmos para a construção de tabelas LALR.

\item Próxima aula: Geradores de analisadores LALR.
\end{itemize}
\section*{Exercícios}
\label{sec:orgab8229e}

\subsection*{Exercícios}
\label{sec:orgb6f0126}

\begin{itemize}
\item Determine se a seguinte gramática possui conflitos,
utilizando o algoritmo de construção de tabelas LALR.
\end{itemize}

\begin{array}{lcl}
E & \to & T \textbf{+} E\,|\,T\\
T & \to & \textbf{x}\\
\end{array}
\end{document}
