% Intended LaTeX compiler: pdflatex
\documentclass[11pt]{article}
\usepackage[utf8]{inputenc}
\usepackage[T1]{fontenc}
\usepackage{graphicx}
\usepackage{longtable}
\usepackage{wrapfig}
\usepackage{rotating}
\usepackage[normalem]{ulem}
\usepackage{amsmath}
\usepackage{amssymb}
\usepackage{capt-of}
\usepackage{hyperref}
\author{Construção de compiladores I}
\date{}
\title{Análise sintática}
\hypersetup{
 pdfauthor={Construção de compiladores I},
 pdftitle={Análise sintática},
 pdfkeywords={},
 pdfsubject={},
 pdfcreator={Emacs 28.2 (Org mode 9.6.7)}, 
 pdflang={English}}
\begin{document}

\maketitle

\section*{Objetivos}
\label{sec:orgdb362a4}

\subsection*{Objetivos}
\label{sec:orgd7dc65e}

\begin{itemize}
\item Introduzir a segunda etapa do front-end: a análise sintática.

\item Introduzir o conceito de analisadores top-down e bottom-up.
\end{itemize}

\subsection*{Objetivos}
\label{sec:org5e286bb}

\begin{itemize}
\item Apresentar a técnica de análise sintática descendente recursiva.
\end{itemize}

\section*{Análise sintática}
\label{sec:org76b2cb6}

\subsection*{Análise sintática}
\label{sec:org11752dd}

\begin{itemize}
\item Responsável por determinar se o programa atende as restrições sintáticas
da linguagem.
\end{itemize}

\subsection*{Análise sintática}
\label{sec:org555344f}

\begin{itemize}
\item Regras sintáticas de uma linguagem são expressas utilizando gramáticas livres de contexto.
\end{itemize}

\subsection*{Análise sintática}
\label{sec:orgbd41fe4}

\begin{itemize}
\item Porque utilizar GLCs e não ERs?
\begin{itemize}
\item ERs não são capazes de representar construções simples de linguagens.
\end{itemize}
\end{itemize}

\subsection*{Análise sintática}
\label{sec:orgc8f3db2}

\begin{itemize}
\item Vamos considerar um fragmento de expressões formado por variáveis, constantes inteiras
adição, multiplicação.
\end{itemize}

\subsection*{Análise sintática}
\label{sec:orgb300ac8}

\begin{itemize}
\item A seguinte ER representa essa linguagem:
\end{itemize}

\begin{array}{c}
base = [a..z]([a..z] | [0..9])^* \\
base((+|*)base)^*
\end{array}

\subsection*{Análise sintática}
\label{sec:org00db93f}

\begin{itemize}
\item A ER anterior aceita palavras como \(a * b + c\).

\item Porém, como determinar a precência entre operadores?
\end{itemize}

\subsection*{Análise sintática}
\label{sec:org23647d7}

\begin{itemize}
\item Podemos usar a precedência usual da aritmética.

\item Porém, não é possível impor uma ordem diferente de avaliação.
\begin{itemize}
\item Para isso, precisamos de parêntesis.
\end{itemize}
\end{itemize}

\subsection*{Análise sintática}
\label{sec:orga14d2c8}

\begin{itemize}
\item Ao incluir parêntesis, temos um problema:
\begin{itemize}
\item Como expressar usando ER que parêntesis estão corretos?
\end{itemize}
\end{itemize}

\subsection*{Análise sintática}
\label{sec:org4fba839}

\begin{itemize}
\item Pode-se provar que a linguagem de parêntesis balanceados não é regular.
\begin{itemize}
\item Usando o lema do bombeamento.
\item Estrutura similar a \(\{0^n1^n\,|\,n\geq 0\}\).
\end{itemize}
\end{itemize}

\subsection*{Análise sintática}
\label{sec:org1124837}

\begin{itemize}
\item Dessa forma, precisamos utilizar GLCs para representar a estrutura sintática
de linguagens.
\end{itemize}

\subsection*{Análise sintática}
\label{sec:orgc5dd19a}

\begin{itemize}
\item Antes de apresentar técnicas de análise sintática, vamos revisar alguns
conceitos sobre GLCs.
\end{itemize}

\section*{Gramáticas Livres de Contexto}
\label{sec:org7a56ee4}

\subsection*{Gramáticas livres de contexto}
\label{sec:org22c0d8b}

\begin{itemize}
\item Uma GLC é \(G=(V,\Sigma,R,P)\), em que
\begin{itemize}
\item \(V\): conjunto de variáveis (não terminais)
\item \(\Sigma\): alfabeto (terminais)
\item \(R \subseteq V\times (V\cup\Sigma)^*\): regras (produções).
\item \(P\in V\): variável de partida.
\end{itemize}
\end{itemize}

\section*{Concluindo}
\label{sec:orgc405299}

\subsection*{Concluindo}
\label{sec:orgf7b371b}

\begin{itemize}
\item Com isso, concluímos o conteúdo de análise léxica.

\item Próxima aula: análise sintática.
\end{itemize}

\section*{Exercícios}
\label{sec:orgb27175d}

\subsection*{Exercícios}
\label{sec:orgfe4ff75}

\begin{itemize}
\item Construa um analisador léxico, utilizando derivadas,  para a linguagem de expressões
aritméticas. Seu analisador deve ler um arquivo, fornecido como argumento de linha
de comando, e imprimir todos os tokens encontrados.
\end{itemize}
\end{document}