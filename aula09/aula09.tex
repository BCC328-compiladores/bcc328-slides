% Intended LaTeX compiler: pdflatex
\documentclass[11pt]{article}
\usepackage[utf8]{inputenc}
\usepackage[T1]{fontenc}
\usepackage{graphicx}
\usepackage{longtable}
\usepackage{wrapfig}
\usepackage{rotating}
\usepackage[normalem]{ulem}
\usepackage{amsmath}
\usepackage{amssymb}
\usepackage{capt-of}
\usepackage{hyperref}
\author{Construção de compiladores I}
\date{}
\title{Conjuntos first e follow}
\hypersetup{
 pdfauthor={Construção de compiladores I},
 pdftitle={Conjuntos first e follow},
 pdfkeywords={},
 pdfsubject={},
 pdfcreator={Emacs 28.2 (Org mode 9.7)}, 
 pdflang={English}}
\begin{document}

\maketitle
\section*{Objetivos}
\label{sec:org8d6932c}

\subsection*{Objetivos}
\label{sec:org01dff8a}

\begin{itemize}
\item Apresentar os conjuntos first e follow e seu uso para definir gramáticas LL(1)

\item Apresentar uma implementação de first e follow em Haskell.
\end{itemize}
\section*{Gramáticas LL(1)}
\label{sec:org205e540}

\subsection*{Gramáticas LL(1)}
\label{sec:org97e6725}

\begin{itemize}
\item Classe de gramáticas que admitem analisadores sem backtracking

\item LL(1) significa
\begin{itemize}
\item *L*eft to right: entrada analisada da esquerda para direita.
\item *L*eft most derivation: construção de derivação mais a esquerda.
\item Uso de \textbf{1} token da entrada para decidir a derivação.
\end{itemize}
\end{itemize}
\subsection*{Gramáticas LL(1)}
\label{sec:orgcad0bd2}

\begin{itemize}
\item Como determinar se uma gramática é LL(1)?
\begin{itemize}
\item Vamos usar os conjuntos First e Follow.
\end{itemize}
\end{itemize}
\section*{First e Follow}
\label{sec:orgea4e52c}

\subsection*{First e Follow}
\label{sec:org66866dc}

\begin{itemize}
\item \(first(\alpha)\): conjunto de terminais que iniciam sentenças derivadas a partir de \(\alpha\).

\item \(\alpha \in (V\cup \Sigma)^*\).
\end{itemize}
\subsection*{First e Follow}
\label{sec:org7660d9c}

\begin{itemize}
\item \(first(a) = \{a\}\), se \(a \in \Sigma\).
\item \(\lambda\in first(A)\) se \(A\to\lambda \in R\).
\end{itemize}
\subsection*{First e Follow}
\label{sec:org351515d}

\begin{itemize}
\item Se \(A \in V\) e
\item \(A \to Y_1\,Y_2\,...\,Y_k \in R\) e
\item \(a\in first(Y_i)\) e
\item \(\forall j. 1 \leq j \leq i - 1. \lambda \in first(Y_j)\) então \(a \in first(A)\).
\end{itemize}
\subsection*{First e Follow}
\label{sec:org726c687}

\begin{itemize}
\item Aplique estas regras enquanto for possível.
\end{itemize}
\subsection*{First e Follow}
\label{sec:orgf079d34}

\begin{itemize}
\item Calcular os conjuntos \(first\) para:
\end{itemize}

\begin{array}{lcl}
E  & \to & TE'\\
E' & \to & \textbf{+}TE'\,|\,\lambda\\
T  & \to & FT' \\
T' & \to & \textbf{*}FT'\,|\,\lambda\\
F  & \to & \textbf{(}E\textbf{)}\,|\,\textbf{id}
\end{array}
\subsection*{First e Follow}
\label{sec:org720a121}

\begin{itemize}
\item \(first(E) = first(T) = first(F)\)

\item \(first(F) = \{\textbf{(}, \textbf{id}\}\)
\end{itemize}
\subsection*{First e Follow}
\label{sec:org8027e64}

\begin{itemize}
\item \(first(E') = \{\textbf{+},\lambda\}\)

\item \(first(T') = \{\textbf{*},\lambda\}\)
\end{itemize}
\subsection*{First e Follow}
\label{sec:org63fcb24}

\begin{itemize}
\item \(follow(A)\): conjunto de terminais que aparecem logo a direita de \(A\) em alguma derivação.

\item \(A \in V\).
\end{itemize}
\subsection*{First e Follow}
\label{sec:org0146466}

\begin{itemize}
\item \$ \(\in follow(P)\)

\item Se \(A \to \alpha B \beta \in R\) então:
\end{itemize}

\begin{array}{l}
first(\beta) - \{\lambda\} \subseteq follow(B)
\end{array}
\subsection*{First e Follow}
\label{sec:orgae2ab56}

\begin{itemize}
\item Se \(A \to \alpha B\) ou \(A \to \alpha B \beta\), em que \(\lambda \in first(\beta)\) então:
\end{itemize}

\begin{array}{l}
follow(A) \subseteq follow(B)
\end{array}
\subsection*{First e Follow}
\label{sec:orgcb0afa8}

\begin{itemize}
\item Aplique estas regras enquanto possível.
\end{itemize}
\subsection*{First e Follow}
\label{sec:orgb7300a0}

\begin{itemize}
\item Calcular os conjuntos \(follow\) para:
\end{itemize}

\begin{array}{lcl}
E  & \to & TE'\\
E' & \to & \textbf{+}TE'\,|\,\lambda\\
T  & \to & FT' \\
T' & \to & \textbf{*}FT'\,|\,\lambda\\
F  & \to & \textbf{(}E\textbf{)}\,|\,\textbf{id}
\end{array}
\subsection*{First e Follow}
\label{sec:org629b3c9}

\begin{itemize}
\item \$ \(\in follow(E)\):
\begin{itemize}
\item \(E\) é a variável inicial da gramática.
\end{itemize}
\end{itemize}
\subsection*{First e Follow}
\label{sec:org724fec5}

\begin{itemize}
\item \textbf{)} \(\in follow(E)\):
\begin{itemize}
\item Devido a produção \(F \to (E)\)
\item Regra: \(A \to \alpha B \beta\) então \(first(\beta) \subseteq follow(B)\).
\begin{itemize}
\item Neste caso, \(\beta = \textbf{)}\)
\end{itemize}
\end{itemize}
\end{itemize}
\subsection*{First e Follow}
\label{sec:org875c47c}

\begin{itemize}
\item Logo, temos que \$follow(E) = $\backslash${\textbf{)},$\backslash$\(\}\).
\end{itemize}
\section*{Implementação em Haskell}
\label{sec:org9be93c9}

\subsection*{Implementação em Haskell}
\label{sec:org755d21d}

\begin{itemize}
\item Representando terminais
\end{itemize}

\begin{verbatim}
data Terminal
  = T String
  | Dollar
  | Lambda
\end{verbatim}
\subsection*{Implementação em Haskell}
\label{sec:orgc526fbf}

\begin{itemize}
\item Representando não terminais
\end{itemize}

\begin{verbatim}
data Nonterminal
  = NT String
\end{verbatim}
\subsection*{Implementação em Haskell}
\label{sec:org512b8fc}

\begin{itemize}
\item Representando símbolos
\end{itemize}

\begin{verbatim}
data Symbol
  = Var Nonterminal
  | Symb Terminal
\end{verbatim}
\subsection*{Implementação em Haskell}
\label{sec:org23dadf7}

\begin{itemize}
\item Representando produções
\end{itemize}

\begin{verbatim}
data Production
  = Prod {
      leftHand :: Nonterminal
    , rightHand :: [Symbol]
    } 
\end{verbatim}
\subsection*{Implementação em Haskell}
\label{sec:orge56ccd3}

\begin{itemize}
\item Representando gramáticas
\end{itemize}

\begin{verbatim}
data Grammar
  = Grammar {
      productions :: [Production]
    , start       :: Nonterminal
    }
\end{verbatim}
\subsection*{Implementação em Haskell}
\label{sec:orgb0e2b15}

\begin{itemize}
\item Representação de ponto fixo.
\end{itemize}

\begin{verbatim}
fixpoint :: Eq a => (a -> a) -> a -> a
fixpoint f x = let x' = f x
               in if x == x' then x
                  else fixpoint f x'
\end{verbatim}
\subsection*{Implementação em Haskell}
\label{sec:org9e43c18}

\begin{itemize}
\item Definição de tabela de first.
\end{itemize}

\begin{verbatim}
type First = Map Nonterminal [Terminal]

merge :: First -> Nonterminal -> [Terminal] -> First
merge m nt ts = Map.insertWith union nt ts m

firstSetFor :: Nonterminal -> First -> [Terminal]
firstSetFor nt m
  = case Map.lookup nt m of
      Nothing -> []
      Just ts -> ts
\end{verbatim}
\subsection*{Implementação em Haskell}
\label{sec:orge6a7272}

\begin{itemize}
\item Cálculo de first
\end{itemize}

\begin{verbatim}
first :: Grammar -> [(Nonterminal, [Terminal])]
first g = Map.toList m
  where
     m = fixpoint (stepFirst g) (first0 g)
\end{verbatim}
\subsection*{Implementação em Haskell}
\label{sec:org6480083}

\begin{itemize}
\item Cálculo de first0
\end{itemize}

\begin{verbatim}
first0 :: Grammar -> First
first0 g = Map.fromList $ map f (nonterminals g)
  where
    f nt = (nt , [])
\end{verbatim}
\subsection*{Implementação em Haskell}
\label{sec:org92abba2}

\begin{itemize}
\item Iteração do conjunto first
\end{itemize}

\begin{verbatim}
stepFirst :: Grammar -> First -> First
stepFirst g current
  = step' (productions g) current
  where
    step' [] curr = curr
    step' (p:ps) curr
      = merge (step' ps curr)
              (leftHand p)
              (terminalsForRHS p)
\end{verbatim}
\subsection*{Implementação em Haskell}
\label{sec:org0a9d1a4}

\begin{itemize}
\item Definição de \texttt{terminalsForRHS}
\end{itemize}

\begin{verbatim}
terminalsForRHS :: Production -> [Terminal]
terminalsForRHS (Prod _ []) = [Lambda]
terminalsForRHS (Prod _ [Symb Lambda]) = [Lambda]
terminalsForRHS (Prod x (yj : ys))
  = case  yj of
      Symb terminal -> [terminal]
      Var nonterminal ->
         if Lambda `elem` first_iminus1 then
            terminalsForYj `union` terminalsForRHS (Prod x ys)
         else
            terminalsForYj
      where
         first_iminus1 = firstSetFor nonterminal current
         terminalsForYj = filter (/= Lambda) first_iminus1
\end{verbatim}
\subsection*{Implementação em Haskell}
\label{sec:org4e7fba5}

\begin{itemize}
\item Extensão de first para palavras
\end{itemize}

\begin{verbatim}
firstForWord :: [Symbol] -> First -> [Terminal]
firstForWord [(Var nt)] ft = firstSetFor nt ft
firstForWord [(Symb t)] _ = [t]
firstForWord ((Var nt) : ss) ft =
  if Lambda `elem` firstSetFor nt ft then 
        firstMinusLambda `union` firstForWord ss ft
  else firstMinusLambda
       where firstMinusLambda = [x | x <- firstSetFor nt ft, x /= Lambda]
firstForWord _ _ = []
\end{verbatim}
\subsection*{Implementação em Haskell}
\label{sec:org06ab902}

\begin{itemize}
\item Implementação de follow
\end{itemize}

\begin{verbatim}
follow :: Grammar -> [(Nonterminal, [Terminal])]
follow g
  = Map.toList $ fixpoint (stepFollow g m) (follow0 g)
  where
    m = Map.fromList $ first g
\end{verbatim}
\subsection*{Implementação em Haskell}
\label{sec:org3bca858}

\begin{itemize}
\item Inicialização de follow
\end{itemize}

\begin{verbatim}
follow0 :: Grammar -> Follow
follow0 g = Map.fromList $ map f (nonterminals g)
  where
    f nt = if nt == start g then (nt, [Dollar]) else (nt, [])
\end{verbatim}
\subsection*{Implementação em Haskell}
\label{sec:org46be6ae}

\begin{itemize}
\item Iteração do conjunto follow
\end{itemize}

\begin{verbatim}
stepFollow :: Grammar -> First -> Follow -> Follow
stepFollow g firstG current
  = step' (productions g) firstG current
    where
      step' [] _ curr = curr
      step' (p : ps) fG curr
        = mergeTerminals p (step' ps fG curr)
       where
         ...
\end{verbatim}
\subsection*{Implementação em Haskell}
\label{sec:org60c15b2}

\begin{itemize}
\item Iteração do conjunto follow
\end{itemize}

\begin{verbatim}
mergeTerminals (Prod _ []) = undefined
mergeTerminals (Prod l (s : ss)) = merge' l [] s ss
\end{verbatim}
\subsection*{Implementação em Haskell}
\label{sec:org139ef9c}

\begin{itemize}
\item Iteração do conjunto follow
\end{itemize}

\begin{verbatim}
merge' a _ (Var b) [] fMinus1
  = merge fMinus1 b (followSetFor a fMinus1)
merge' _ _  (Symb _) [] fMinus1 = fMinus1
merge' a w1 (Symb t) (w2 : w2s) fMinus1
 = merge' a (w1 ++ [Symb t]) w2 w2s fMinus1
merge' a w1 nt@(Var b) w2@(w21 : w2s) fMinus1
  = merge (merge' a (w1 ++ [nt]) w21 w2s fMinus1) b new
    where
       firstW2 = [x | x <- firstForWord w2 fG, x /= Lambda]
       new = if Lambda `elem` firstForWord w2 fG then
                followSetFor a fMinus1 `union` firstW2
              else firstForWord w2 fG
\end{verbatim}
\section*{Concluindo}
\label{sec:orgee6d676}

\subsection*{Concluindo}
\label{sec:org18baf81}

\begin{itemize}
\item Nesta aula, apresentamos os conjuntos first e follow.

\item Apresentamos uma implementação simples em Haskell de funções para calcular estes conjuntos.
\end{itemize}
\subsection*{Concluindo}
\label{sec:org828a3a2}

\begin{itemize}
\item Próxima aula: Análise sintática LL(1)
\end{itemize}
\section*{Exercícios}
\label{sec:org438f7a3}

\subsection*{Exercícios}
\label{sec:orgd183041}

\begin{itemize}
\item Utilizando a representação de gramáticas utilizada para obter os conjuntos first e follow,
implemente uma função para obter o conjunto de não terminais anuláveis de uma gramática.
Dizemos que um não terminal é anulável se ele deriva \(\lambda\) em um ou mais passos de derivação.
\end{itemize}
\end{document}
