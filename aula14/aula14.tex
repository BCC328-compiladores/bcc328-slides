% Intended LaTeX compiler: pdflatex
\documentclass[11pt]{article}
\usepackage[utf8]{inputenc}
\usepackage[T1]{fontenc}
\usepackage{graphicx}
\usepackage{longtable}
\usepackage{wrapfig}
\usepackage{rotating}
\usepackage[normalem]{ulem}
\usepackage{amsmath}
\usepackage{amssymb}
\usepackage{capt-of}
\usepackage{hyperref}
\author{Construção de compiladores I}
\date{}
\title{Análise LR(1)}
\hypersetup{
 pdfauthor={Construção de compiladores I},
 pdftitle={Análise LR(1)},
 pdfkeywords={},
 pdfsubject={},
 pdfcreator={Emacs 28.2 (Org mode 9.7)}, 
 pdflang={English}}
\begin{document}

\maketitle
\section*{Objetivos}
\label{sec:org82033a6}

\subsection*{Objetivos}
\label{sec:orge8c2a2a}

\begin{itemize}
\item Apresentar o algoritmo de análise sintática LR(1).
\end{itemize}
\section*{Introdução}
\label{sec:orgc3503d1}

\subsection*{Introdução}
\label{sec:orgc1b70f3}

\begin{itemize}
\item Nas aulas anteriores, vimos o as funções de fechamento e goto,
utilizadas para a construção de itens LR(0), do autômato LR(0)
e do algoritmo de análise sintática LR(0) e SLR.
\end{itemize}
\subsection*{Introdução}
\label{sec:org561468a}

\begin{itemize}
\item Nesta aula, vamos conhecer outro algoritmo de análise
ascendente: LR(1)

\item O algoritmo LR(1) será um algoritmo que usará o conceito de \textbf{lookahead}.
\end{itemize}
\subsection*{Introdução}
\label{sec:orgaafc30e}

\begin{itemize}
\item \textbf{Lookahead}: primeiro símbolo da entrada, ainda não processado pelo algoritmo.
\end{itemize}
\subsection*{Introdução}
\label{sec:org03444a3}

\begin{itemize}
\item Um item LR(1) é formado por uma produção, contendo um marcador, e um símbolo de lookahead.

\item Se \(A \to \alpha\) é uma regra, então \([A \to .\alpha, a]\) é um item e \(a\) é o lookahead.
\end{itemize}
\section*{Construção dos itens LR(1)}
\label{sec:org937a90d}

\subsection*{Construção dos itens LR(1)}
\label{sec:org17896cc}

\begin{itemize}
\item Fechamento de conjunto de itens \(I\).
\begin{itemize}
\item \(I\subseteq closure(I)\).
\item Para cada item \([A\to \alpha \textbf{.}B\beta,a]\) em \(I\)
\begin{itemize}
\item Para cada regra \(B \to \gamma\) em \(G'\)
\begin{itemize}
\item Para cada \(b\in first(\beta a)\)
\begin{itemize}
\item Adicione \([B \to .\gamma,b]\) em \(I\)
\end{itemize}
\end{itemize}
\end{itemize}
\end{itemize}
\item Repita enquanto houver alterações em \(I\).
\end{itemize}
\subsection*{Construção dos itens LR(1)}
\label{sec:org519263b}

\begin{itemize}
\item Função de \(goto(I,X)\)
\begin{itemize}
\item Inicialize \(J\) como \(\emptyset\).
\item Para cada item \([A\to \alpha .X \beta,a]\) em \(I\)
\begin{itemize}
\item Adicione o item \([A \to \alpha X. \beta, a]\) ao conjunto \(J\).
\end{itemize}
\item retorne \(closure(J)\)
\end{itemize}
\end{itemize}
\subsection*{Construção dos itens LR(1)}
\label{sec:orga59a8a3}

\begin{itemize}
\item Função de construção de itens \(G'\)
\begin{itemize}
\item inicializa \(C\) como closure(\{[S \(\to\) .S, \$]\})
\begin{itemize}
\item Para cada conjunto \(I \in C\)
\begin{itemize}
\item Para cada símbolo \(X\) de \(G'\)
\begin{itemize}
\item se \(goto(I,X) \neq \emptyset \land goto(I,X) \not\in C\)
\begin{itemize}
\item Adicione \(goto(I,X)\) em \(C\)
\end{itemize}
\end{itemize}
\end{itemize}
\end{itemize}
\item repetir enquanto houver alterações em \(C\).
\end{itemize}
\end{itemize}
\section*{Construção da tabela LR(1)}
\label{sec:org909a881}

\subsection*{Construção da tabela LR(1)}
\label{sec:org998ffa4}

\begin{itemize}
\item Se \([A \to \alpha .a\beta,b] \in I_i\) e \(goto(I_i,a) = I_j\),
\begin{itemize}
\item A[i,a] = shift j.
\end{itemize}
\end{itemize}
\subsection*{Construção da tabela LR(1)}
\label{sec:orgfe7e447}

\begin{itemize}
\item Se \([A \to \alpha . , a] \in I_i\) e \(A \neq S'\)
\begin{itemize}
\item A[i,a] = reduce A \(\to\) \(\alpha\)
\end{itemize}
\end{itemize}
\subsection*{Construção da tabela LR(1)}
\label{sec:org4c6b260}

\begin{itemize}
\item Se [S' \(\to\) S., \$] \(\in\) I\textsubscript{i}
\begin{itemize}
\item A[i,\$] = accept
\end{itemize}
\end{itemize}
\subsection*{Construção da tabela LR(1)}
\label{sec:org41889fa}

\begin{itemize}
\item Se goto(I\textsubscript{i,A}) = I\textsubscript{j} então G[i,A] = j
\end{itemize}
\section*{Exemplo}
\label{sec:orgc25450d}

\subsection*{Exemplo}
\label{sec:org0dba1c3}

\begin{itemize}
\item Construção da tabela para a gramática
\end{itemize}

\begin{array}{lcl}
   S & \to & ( L ) \,|\, x\\
   L & \to & L , S \,|\, S\\
\end{array}
\section*{Concluindo}
\label{sec:orga4b9e18}

\subsection*{Concluindo}
\label{sec:orgb95497a}

\begin{itemize}
\item Nesta aula apresentamos a construção de tabelas LR(1).

\item Próxima aula: Analisadores sintáticos LALR.
\end{itemize}
\section*{Exercícios}
\label{sec:org55b76be}

\subsection*{Exercícios}
\label{sec:orgc7e0788}

\begin{itemize}
\item Determine se a seguinte gramática possui conflitos,
utilizando o algoritmo de construção de tabelas LR(1).
\end{itemize}

\begin{array}{lcl}
E & \to & T \textbf{+} E\,|\,T\\
T & \to & \textbf{x}\\
\end{array}
\end{document}
