% Intended LaTeX compiler: pdflatex
\documentclass[11pt]{article}
\usepackage[utf8]{inputenc}
\usepackage[T1]{fontenc}
\usepackage{graphicx}
\usepackage{longtable}
\usepackage{wrapfig}
\usepackage{rotating}
\usepackage[normalem]{ulem}
\usepackage{amsmath}
\usepackage{amssymb}
\usepackage{capt-of}
\usepackage{hyperref}
\author{Construção de compiladores I}
\date{}
\title{Geradores de analisadores LALR}
\hypersetup{
 pdfauthor={Construção de compiladores I},
 pdftitle={Geradores de analisadores LALR},
 pdfkeywords={},
 pdfsubject={},
 pdfcreator={Emacs 28.2 (Org mode 9.7)}, 
 pdflang={English}}
\begin{document}

\maketitle
\section*{Objetivos}
\label{sec:org790d602}

\subsection*{Objetivos}
\label{sec:org755955f}

\begin{itemize}
\item Apresentar o gerador de analisadores sintáticos LALR para Haskell.
\end{itemize}
\section*{Introdução}
\label{sec:org66f10e4}

\subsection*{Introdução}
\label{sec:orgd1775ba}

\begin{itemize}
\item Nas aulas anteriores, vimos alguns algoritmos LR.
\begin{itemize}
\item Em especial, o LALR.
\end{itemize}

\item Este algoritmo consegue realizar o parsing de construções
de linguagens de programação.
\end{itemize}
\subsection*{Introdução}
\label{sec:org100d5db}

\begin{itemize}
\item Nesta aula, veremos uma ferramenta para produzir estes analisadores automaticamente.

\item Este tipo de ferramenta é disponível em praticamente todas as linguagens de programação.
\end{itemize}
\subsection*{Introdução}
\label{sec:org1ab801e}

\begin{itemize}
\item Em Haskell, o analisador que utilizaremos é o Happy, capaz de produzir analisadores LALR a partir de gramáticas.
\end{itemize}
\subsection*{Introdução}
\label{sec:orged6b48a}

\begin{itemize}
\item Apresentaremos a linguagem do Happy, utilizando uma pequena linguagem de expressões.
\end{itemize}
\section*{Especificações Happy}
\label{sec:orgb5852b4}

\subsection*{Especificações Happy}
\label{sec:orgc766b4e}

\begin{itemize}
\item Especificações são formadas por:
\begin{itemize}
\item Trechos de código Haskell
\item Declarações de tokens, erro e nome do parser
\item Definição da gramática.
\end{itemize}
\end{itemize}
\subsection*{Especificações Happy}
\label{sec:orgfe4682c}

\begin{itemize}
\item Trechos de código Haskell são usados para:
\begin{itemize}
\item Definir funções auxiliares.
\item Definir tipos de dados a serem usados pelo parser.
\end{itemize}
\end{itemize}
\subsection*{Especificações Happy}
\label{sec:org4751114}

\begin{itemize}
\item Declarações
\end{itemize}

\begin{verbatim}
 expParser
%tokentype { Token }
%error { parseError }
\end{verbatim}
\subsection*{Especificações Happy}
\label{sec:org09d2062}

\begin{itemize}
\item Definições de tokens
\end{itemize}

\begin{verbatim}
%token
     int  {TNumber $$}
     var  {TVar $$}
     '+'  {TAdd}
     '*'  {TMul}
     '('  {TLParen}
     ')'  {TRParen}
\end{verbatim}
\subsection*{Especificações Happy}
\label{sec:org305efda}

\begin{itemize}
\item Definição da gramática
\end{itemize}

\begin{verbatim}
Expr   : Term '+' Expr   {Add $1 $3}
       | Term            {$1}

Term   : Factor '*' Term {Mul $1 $3}
       | Factor          {$1}

Factor : int             {Number $1}
       | var             {Var $1}
       | '(' Expr ')'    {$2}
\end{verbatim}
\subsection*{Especificações Happy}
\label{sec:org34a5cfe}

\begin{itemize}
\item Podemos produzir um parser LALR a partir da especificação mostrada.
\end{itemize}

\begin{verbatim}
happy Parser.y -o Parser.hs
\end{verbatim}
\subsection*{Especificações Happy}
\label{sec:org975284d}

\begin{itemize}
\item Produzindo o conjunto de itens
\begin{itemize}
\item Produzindo arquivo Parser.info
\end{itemize}
\end{itemize}

\begin{verbatim}
happy -i Parser.y
\end{verbatim}
\subsection*{Especificações Happy}
\label{sec:org3ee4795}

\begin{itemize}
\item Exemplo de gramática com conflitos de shift/reduce

\item Solucionando o conflito com predências.
\end{itemize}
\section*{Mônadas}
\label{sec:orgdd561a7}

\subsection*{Mônadas}
\label{sec:orgee0464f}

\begin{itemize}
\item Para construir parsers com suporte
\begin{itemize}
\item Manter contagem da linha atual.
\item Melhor diagnóstico de erros.
\end{itemize}
\item Precisamos de mônadas
\end{itemize}
\subsection*{Mônadas}
\label{sec:orgd7875d9}

\begin{itemize}
\item Tanto o Happy quanto o Alex possuem suporte a mônadas para lidar com estado (número de linhas e colunas).

\item Vamos iniciar com mudanças na especificação Alex.
\end{itemize}
\subsection*{Mônadas}
\label{sec:org2bf99a6}

\begin{itemize}
\item Inserimos suporte a mônadas em especificações Alex usando o wrapper monadUserState

\item Com isso, ações para produzir tokens passam a ter o tipo
\begin{itemize}
\item Tipo Alex: mônada usada pelo analisador léxico.
\end{itemize}
\end{itemize}

\begin{verbatim}
type Action a = AlexInput -> Int -> Alex a
\end{verbatim}
\subsection*{Mônadas}
\label{sec:org5153a9b}

\begin{itemize}
\item Tipo AlexInput
\end{itemize}

\begin{verbatim}
type AlexInput =
  ( AlexPosn      -- current position,
  , Char          -- previous char
  , [Byte]        -- rest of the bytes for the current char
  , String        -- current input string
  )
\end{verbatim}
\subsection*{Mônadas}
\label{sec:org5ddc3a2}

\begin{itemize}
\item Tipo AlexPosn
\end{itemize}

\begin{verbatim}
data AlexPosn = AlexPn
  !Int            -- absolute character offset
  !Int            -- line number
  !Int            -- column number
\end{verbatim}
\subsection*{Mônadas}
\label{sec:org7ec5941}

\begin{itemize}
\item Produzindo tokens
\end{itemize}

\begin{verbatim}
tokenInteger :: Action RangedToken
tokenInteger inp@(_, _, _, str) len
  = pure RangedToken {
           token_ = TNumber $ read $ take len str
         , range_ = mkRange inp len
         }
\end{verbatim}
\subsection*{Mônadas}
\label{sec:org6c29206}

\begin{itemize}
\item Tipo RangedToken
\end{itemize}

\begin{verbatim}
data Range
  = Range {
      start_ :: AlexPosn
    , stop_  :: AlexPosn
    }
data RangedToken
  = RangedToken {
      token_ :: Token
    , range_ :: Range
    } deriving (Eq, Show)
\end{verbatim}
\subsection*{Mônadas}
\label{sec:orgef797c6}

\begin{itemize}
\item Com isso, resolvemos o problema de produzir informações de posições fazendo com que o analisador léxico retorne RangedToken's.
\end{itemize}
\subsection*{Mônadas}
\label{sec:org5de3709}

\begin{itemize}
\item Outro problema: como permitir comentários de várias linhas?
\begin{itemize}
\item Comentários aninhados?
\end{itemize}
\end{itemize}
\subsection*{Mônadas}
\label{sec:org0fdef86}

\begin{itemize}
\item Para lidar com isso, vamos criar um estado para armazenar o nível de comentários aninhados.
\begin{itemize}
\item Tipo deve possuir esse nome
\end{itemize}
\end{itemize}

\begin{verbatim}
data AlexUserState
  = AlexUserState {
      level :: Int
    }
\end{verbatim}
\subsection*{Mônadas}
\label{sec:org12a5611}

\begin{itemize}
\item Funções da mônada
\end{itemize}

\begin{verbatim}
get :: Alex AlexUserState
get = Alex $ \s -> Right (s, alex_ust s)

put :: AlexUserState -> Alex ()
put s' = Alex $ \s -> Right (s{alex_ust = s'}, ())

modify :: (AlexUserState -> AlexUserState) -> Alex ()
modify f = Alex $ \s -> Right (s{alex_ust = f (alex_ust s)}, ())
\end{verbatim}
\subsection*{Mônadas}
\label{sec:org80bf425}

\begin{itemize}
\item Funções para lidar com aninhamento
\end{itemize}

\begin{verbatim}
nestComment :: Action RangedToken
nestComment input len = do
  modify $ \s -> s{level = level s + 1}
  skip input len
\end{verbatim}
\subsection*{Mônadas}
\label{sec:org3f80a4f}

\begin{itemize}
\item Funções para lidar com aninhamento
\end{itemize}

\begin{verbatim}
unnestComment :: Action RangedToken
unnestComment input len = do
  state <- get
  let level' = level state - 1
  put state{level = level'}
  when (level' == 0) $
    alexSetStartCode 0
  skip input len
\end{verbatim}
\subsection*{Mônadas}
\label{sec:orgd49914c}

\begin{itemize}
\item Definição do estado inicial
\end{itemize}

\begin{verbatim}
alexInitUserState :: AlexUserState
alexInitUserState = AlexUserState 0
\end{verbatim}
\subsection*{Mônadas}
\label{sec:orgf992621}

\begin{itemize}
\item Definição de tokens com marcações de estados
\end{itemize}

\begin{verbatim}
tokens :-
<0>  $white+                 ; -- removing whitespace
<0>  @number                 { tokenInteger }
<0>  @identifier             { tokenId }
<0>  \+                      { simpleToken TAdd }
...
<0>  "{-"                    { nestComment `andBegin` comment }
<0>  "-}"                    { \ _ _ -> alexError "Error: unexpected close comment!" }
\end{verbatim}
\subsection*{Mônadas}
\label{sec:org4681a10}

\begin{itemize}
\item Definição de tokens com marcações de estados
\end{itemize}

\begin{verbatim}
<comment> "{-"               { nestComment }
<comment> "-}"               { unnestComment }
<comment> .                  ;
<comment> \n                 ;
\end{verbatim}
\subsection*{Mônadas}
\label{sec:orgb3869bd}

\begin{itemize}
\item Mudanças na especificação Happy.
\end{itemize}
\subsection*{Mônadas}
\label{sec:org52b22d0}

\begin{itemize}
\item Definição de uma mônada
\end{itemize}

\begin{verbatim}
%monad { Alex }{ >>= }{ pure }
%lexer { lexer }{ RangedToken EOF _ }
\end{verbatim}
\subsection*{Mônadas}
\label{sec:org4787bf9}

\begin{itemize}
\item Especificação dos tokens
\end{itemize}

\begin{verbatim}
%token
     int  {RangedToken (TNumber _) _}
     var  {RangedToken (TVar _) _}
     '+'  {RangedToken TAdd _}
     '*'  {RangedToken TMul _}
     '('  {RangedToken TLParen _}
     ')'  {RangedToken TRParen _}
\end{verbatim}
\subsection*{Mônadas}
\label{sec:org910f8ee}

\begin{itemize}
\item Combinando posições de linha
\end{itemize}

\begin{verbatim}
(<->) :: Range -> Range -> Range
(Range a1 _) <-> (Range _ b2) = Range a1 b2
\end{verbatim}
\subsection*{Mônadas}
\label{sec:orgecf5846}

\begin{itemize}
\item Especificação da gramática
\end{itemize}

\begin{verbatim}
Expr   : Expr '+' Expr   {Add ((info $1) <-> (info $3)) $1 $3}
Expr   : Expr '*' Expr   {Mul ((info $1) <-> (info $3)) $1 $3}
       | int             {Number (info $1) (unNumber $1)}
       | var             {Var (info $1) (unId $1)}
       | '(' Expr ')'    {$2}
\end{verbatim}
\section*{Concluindo}
\label{sec:org44fb423}

\subsection*{Concluindo}
\label{sec:orgb761d23}

\begin{itemize}
\item Nesta aula apresentamos o gerador de analisadores sintáticos Happy.

\item Mostramos como adicionar suporte a marcação de linhas / colunas.
\end{itemize}
\subsection*{Concluindo}
\label{sec:orgc700c66}

\begin{itemize}
\item Próximas aulas: Semântica formal e interpretadores.
\end{itemize}
\section*{Exercícios}
\label{sec:org4c523cc}

\subsection*{Exercícios}
\label{sec:orgc157b51}

\begin{itemize}
\item Utilizando o gerador de analisadores sintáticos Happy, construa um analisador sintático para sintaxe de fórmulas da lógica proposicional.
\end{itemize}
\end{document}
