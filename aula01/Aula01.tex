% Intended LaTeX compiler: pdflatex
\documentclass[11pt]{article}
\usepackage[utf8]{inputenc}
\usepackage[T1]{fontenc}
\usepackage{graphicx}
\usepackage{longtable}
\usepackage{wrapfig}
\usepackage{rotating}
\usepackage[normalem]{ulem}
\usepackage{amsmath}
\usepackage{amssymb}
\usepackage{capt-of}
\usepackage{hyperref}
\author{Construção de compiladores I}
\date{}
\title{Revisão de Haskell}
\hypersetup{
 pdfauthor={Construção de compiladores I},
 pdftitle={Revisão de Haskell},
 pdfkeywords={},
 pdfsubject={},
 pdfcreator={Emacs 28.2 (Org mode 9.7)}, 
 pdflang={English}}
\begin{document}

\maketitle
\section*{Objetivos}
\label{sec:org21b0edc}

\subsection*{Objetivos}
\label{sec:org1b4ca65}

\begin{itemize}
\item Revisar conceitos de programação funcional utilizando Haskell.

\item Apresentar a aplicação dos conceitos para construir um compilador protótipo.
\end{itemize}
\section*{Markdown}
\label{sec:org93740e3}

\subsection*{Markdown}
\label{sec:org3cf2e35}

\begin{itemize}
\item Formato amplamente utilizado para representação de documentos.

\item Popularizado pela adoção de plataformas como Github e 
ferramentas como o pandoc.
\end{itemize}
\subsection*{Markdown}
\label{sec:orga6f8a7a}

\begin{itemize}
\item Componentes do compilador.
\begin{itemize}
\item Analisadores léxico e sintático: produção da AST do formato.

\item Gerador de código: produção de HTML.
\end{itemize}
\end{itemize}
\subsection*{Markdown}
\label{sec:org1239a06}

\begin{itemize}
\item Primeiro passo: criar uma biblioteca para criação de HTML bem formado.
\item Faremos isso criando uma EDSL para HTML
\end{itemize}
\section*{EDSLs}
\label{sec:org0ff6aae}

\subsection*{EDSLs}
\label{sec:org2d226f9}

\begin{itemize}
\item Em Haskell, é bem comum desenvolvermos EDSLs
\begin{itemize}
\item \textbf{E} mbedded \textbf{D} omain \textbf{S} pecific \textbf{L} anguage
\end{itemize}
\item DSLs estão em toda parte
\begin{itemize}
\item makefiles, sql, dot, html\ldots{}
\end{itemize}
\end{itemize}
\subsection*{EDSLs}
\label{sec:org9041c8d}

\begin{itemize}
\item Mas porque ``Embedded''?
\begin{itemize}
\item EDSLs utilizam todo os recursos da linguagem de origem.
\item Vantagem: não é necessário construir um compilador do ``zero''.
\end{itemize}
\end{itemize}
\subsection*{EDSLs}
\label{sec:orgfbe8ada}

\begin{itemize}
\item Mas qual o problema que a EDSL de Html vai resolver?

\item A princípio, podemos produzir Html usando operações sobre strings\ldots{}
\end{itemize}
\subsection*{EDSLs}
\label{sec:org024679f}

\begin{itemize}
\item Problemas de uso de strings: 
\begin{itemize}
\item Quem garante que construímos <title><\title> somente dentro 
de um <head> usando strings?
\end{itemize}
\end{itemize}
\subsection*{EDSLs}
\label{sec:org0d9d66b}

\begin{itemize}
\item Estratégia de projeto: Uma biblioteca de combinadores
\begin{itemize}
\item Funções para elementos primitivos.
\item Funções para combinar elementos HTML
\end{itemize}
\end{itemize}
\section*{EDSL para Html}
\label{sec:org3d8a48e}

\subsection*{EDSL para Html}
\label{sec:orgce9d56f}

\begin{itemize}
\item Definição de tipos:
\begin{itemize}
\item Tipos encapsulam strings correspondente a estrutura de HTML.
\end{itemize}
\end{itemize}
\begin{verbatim}
newtype Html
  = Html String

newtype Structure
  = Structure String

newtype Content
  = Content String

newtype Head
  = Head String
\end{verbatim}
\subsection*{EDSL para HTML}
\label{sec:org938b239}

\begin{itemize}
\item Definindo um novo tipo
\begin{itemize}
\item \texttt{newtype} permite definir tipos com um único construtor.
\end{itemize}
\end{itemize}

\begin{verbatim}
newtype Html
  = Html String
\end{verbatim}
\subsection*{EDSL para HTML}
\label{sec:orgca58563}

\begin{itemize}
\item Construtores são funções.
\end{itemize}

\begin{verbatim}
Html :: String -> Html
\end{verbatim}
\subsection*{EDSL para HTML}
\label{sec:org0d3b88e}

\begin{itemize}
\item Gerando o código HTML:
\begin{itemize}
\item Uso de casamento de padrão
\end{itemize}
\end{itemize}

\begin{verbatim}
render :: Html -> String
render (Html str) = str
\end{verbatim}
\subsection*{EDSL para HTML}
\label{sec:org76e739c}

\begin{itemize}
\item Qual a vantagem disso? Porque não strings?
\begin{itemize}
\item Vamos ``ocultar'' a estrutura de tipos.
\item Manipulação apenas por funções.
\item Funções garantem a geração de HTML válido.
\end{itemize}
\end{itemize}
\subsection*{EDSL para Html}
\label{sec:org426c961}

\begin{itemize}
\item Criando uma tag qualquer:
\end{itemize}

\begin{verbatim}
el :: String -> String -> String
el tag content =
  "<" <> tag <> ">" <> content <> "</" <> tag <> ">"
\end{verbatim}
\subsection*{EDSL para Html}
\label{sec:org281788c}

\begin{itemize}
\item Mas o que é a função \texttt{<>}?
\end{itemize}

\begin{verbatim}
class Semigroup a where
  (<>) :: a -> a -> a

class Semigroup a => Monoid a where
  mempty :: a
\end{verbatim}
\subsection*{EDSL para Html}
\label{sec:orgdeae22e}

\begin{itemize}
\item \texttt{Semigroup} e \texttt{Monoid} são classes de tipos 
\begin{itemize}
\item Duas operações definidas: \texttt{(<>)} e \texttt{mempty}.
\end{itemize}
\item Requisitos
\begin{itemize}
\item \texttt{(<>)} deve ser associativo.
\item \texttt{mempty} deve ser elemento neutro de \texttt{(<>)}.
\end{itemize}
\end{itemize}
\subsection*{EDSL para Html}
\label{sec:org64f0553}

\begin{itemize}
\item Listas são instância de \texttt{Semigroup} e \texttt{Monoid}:
\begin{itemize}
\item Strings são apenas listas de caracteres.
\end{itemize}
\end{itemize}

\begin{verbatim}
instance Semigroup [a] where
  xs <> ys = xs ++ ys

instance Monoid [a] where
   mempty        = []
\end{verbatim}
\subsection*{EDSL para Html}
\label{sec:org2073b86}

\begin{itemize}
\item A partir da função \texttt{el} podemos criar outras tags.

\item Código para definir a tag \texttt{<b> ... </b>}:
\end{itemize}

\begin{verbatim}
getContentString :: Content -> String
getContentString (Content str) = str

b_ :: Content -> Content
b_ content =
  Content $ el "b" (getContentString content)
\end{verbatim}
\subsection*{EDSL para Html}
\label{sec:org22b508c}

\begin{itemize}
\item Demais tags seguem estrutura similar.
\end{itemize}

\begin{verbatim}
i_ :: Content -> Content
i_ content =
  Content $ el "i" (getContentString content)
\end{verbatim}
\subsection*{EDSL para HTML}
\label{sec:org1a65271}

\begin{itemize}
\item Como usar essa DSL para construir HTML?
\end{itemize}
\subsection*{EDSL para HTML}
\label{sec:orgc938e2f}

\begin{itemize}
\item Construir esse HTML simples:
\end{itemize}

\begin{verbatim}
  <html>
    <head>
      <title>My title</title>
    </head>
    <body>
      <h1>Heading</h1>
      <p>Paragraph #1</p>
      <p>Paragraph #2</p>
    </body>
  </html>
\end{verbatim}
\subsection*{EDSL para HTML}
\label{sec:org4a62fe5}

\begin{itemize}
\item Representação na EDSL:
\end{itemize}

\begin{verbatim}
myhtml :: Html
myhtml =
  html_
    (title_ "My title")
    ((h_ 1 (txt_ "Heading"))    <>
     (p_ $ txt_ "Paragraph #1") <>
     (p_ $ txt_ "Paragraph #2"))
\end{verbatim}
\subsection*{EDSL para HTML}
\label{sec:org02cfd75}

\begin{itemize}
\item Elementos da DSL são combinados utilizando \texttt{Monoid}.
\end{itemize}

\begin{verbatim}
instance Semigroup Head where
  (Head h1) <> (Head h2) = Head (h1 <> h2)

instance Monoid Head where
  mempty = Head ""
\end{verbatim}
\subsection*{EDSL para HTML}
\label{sec:orgbc24c74}

\begin{itemize}
\item Tipos \texttt{Structure} e \texttt{Content} também são monóides.
\end{itemize}
\subsection*{EDSL para HTML}
\label{sec:orgc95e429}

\begin{itemize}
\item Definida no módulo \texttt{Markup.Printer.Html.Internal}.

\item Esse módulo é encapsulado por \texttt{Markup.Printer.Html} que
exporta apenas funções para manipular os tipos \texttt{Html, Structure, Content}.
\end{itemize}
\subsection*{EDSL para HTML}
\label{sec:org0757a19}

\begin{itemize}
\item Definição do cabeçalho
\end{itemize}

\begin{verbatim}
module Markup.Printer.Html ( Html
                           , Head
                           , title_
                           , Structure
                           , html_
                           , h_
                           , p_
                           , ul_
                           , code_
                           , Content
                           , txt_
                           , b_
                           , i_
                           , render
                           )  where

import Markup.Printer.Html.Internal
\end{verbatim}
\subsection*{EDSL para HTML}
\label{sec:org7d77a49}

\begin{itemize}
\item Seguindo esse padrão, manipulação de valores \texttt{HTML} só
pode ser feito usando funções da EDSL.
\end{itemize}
\section*{De Markdown para HTML}
\label{sec:org18f85e0}

\subsection*{De Markdown para HTML}
\label{sec:org105adda}

\begin{itemize}
\item Agora que representamos HTML, vamos modelar a entrada do compilador.
\end{itemize}
\subsection*{De Markdown para HTML}
\label{sec:org54a9b4f}

\begin{itemize}
\item Primeiro, devemos representar a estrutura de Markdown como um tipo
Haskell
\end{itemize}
\subsection*{De Markdown para HTML}
\label{sec:orgcc4ece1}

\begin{itemize}
\item Documentos são listas de estruturas.
\end{itemize}

\begin{verbatim}
type Document = [Structure]
\end{verbatim}
\subsection*{De Markdown para HTML}
\label{sec:org5261a95}

\begin{itemize}
\item Estruturas são dos seguintes tipos:
\begin{itemize}
\item Cabeçalhos
\item Parágrafos
\item Listas não ordenadas
\item Código fonte
\end{itemize}
\end{itemize}
\subsection*{De Markdown para HTML}
\label{sec:org2c593f2}

\begin{itemize}
\item Representação em Haskell
\end{itemize}

\begin{verbatim}
data Structure
  = Heading Natural String
  | Paragraph String
  | UnorderedList [String] 
  | CodeBlock [String]
  deriving Show
\end{verbatim}
\subsection*{De Markdown para HTML}
\label{sec:org368c072}

\begin{itemize}
\item Exemplos de representação: cabeçalhos e parágrafos.
\end{itemize}

\begin{verbatim}
* BCC328 - Construção de compiladores I

Revisão de Haskell
\end{verbatim}
\subsection*{De Markdown para HTML}
\label{sec:org0e07c34}

\begin{itemize}
\item Representação em Haskell:
\end{itemize}

\begin{verbatim}
example1 :: Document
example1 =
  [ Heading 1 "BCC328 - Construção de compiladores I"
  , Paragraph "Revisão de Haskell"
  ]
\end{verbatim}
\subsection*{De Markdown para HTML}
\label{sec:orgbadb732}

\begin{itemize}
\item Exemplos de representação: Listas e código fonte.
\end{itemize}

\begin{verbatim}
- Como compilar código Haskell:
> ➜ ghc Main.hs
> [1 of 1] Compiling Main ( Main.hs, Main.o )
> Linking Main ...
Com isso o GHC irá criar os seguintes arquivos:
- Main.hi: Arquivo de interface.
- Main.o: Código objeto antes de link-edição.
- Main (ou Main.exe no Windows): executável
\end{verbatim}
\subsection*{De Markdown para HTML}
\label{sec:org6f80024}

\begin{itemize}
\item Representação em Haskell:
\end{itemize}

\begin{verbatim}
example2 :: Document
example2 =
  [ UnorderedList ["Como compilar código Haskell:"]
  , CodeBlock [ "➜ ghc Main.hs"
              , "[1 of 1] Compiling Main ( Main.hs, Main.o )"
              , "Linking Main ..."]
  , Paragraph "Com isso o GHC irá criar os seguintes arquivos:"
  , UnorderedList [ "Main.hi: Arquivo de interface."
                  , "Main.o: Código objeto antes de link-edição."
                  , "Main (ou Main.exe no Windows): executável"]
  ]
\end{verbatim}
\subsection*{De Markdown para HTML}
\label{sec:org82db6fe}

\begin{itemize}
\item Gerando HTML a partir de um documento.
\end{itemize}

\begin{verbatim}
translate :: AST.Document -> HTML.Structure
translate = foldMap translateStructure

foldMap :: Monoid m => (a -> m) -> [a] -> m
foldMap f = foldr ((<>) . f) mempty
\end{verbatim}
\subsection*{De Markdown para HTML}
\label{sec:org1f0fb1d}

\begin{itemize}
\item Gerando HTML a partir de estruturas.
\end{itemize}

\begin{verbatim}
import qualified Markup.Language.Syntax as AST
import qualified Markup.Printer.Html as HTML

translateStructure :: AST.Structure -> HTML.Structure
translateStructure (AST.Heading n txt)
  = HTML.h_ n $ HTML.txt_ txt
translateStructure (AST.Paragraph p)
  = HTML.p_ $ HTML.txt_ p
translateStructure (AST.UnorderedList ds)
  = HTML.ul_ $ map (HTML.p_ . HTML.txt_) ds
translateStructure (AST.CodeBlock ds)
  = HTML.code_ (unlines ds)
\end{verbatim}
\section*{Testes}
\label{sec:org1806986}

\subsection*{Testes}
\label{sec:orgc92d454}

\begin{itemize}
\item Infraestrutura de golden tests disponível.

\item Golden tests
\begin{itemize}
\item Entradas disponíveis como arquivos.
\item Saídas esperadas disponíveis como arquivos.
\end{itemize}
\end{itemize}
\subsection*{Testes}
\label{sec:orgfb200ac}

\begin{itemize}
\item Execução dos testes
\begin{itemize}
\item Após iniciar o \texttt{nix-shell}, execute:
\end{itemize}
\end{itemize}

\begin{verbatim}
cabal new-test
\end{verbatim}
\section*{Concluindo}
\label{sec:org10a49c6}

\subsection*{Concluindo}
\label{sec:org1799337}

\begin{itemize}
\item Nesta aula revisamos alguns conceitos da linguagem Haskell:
\begin{itemize}
\item Definição de tipos de dados.
\item Definição de funções por casamento de padrão.
\item Funções de ordem superior
\end{itemize}
\end{itemize}
\subsection*{Concluindo}
\label{sec:orgc270251}

\begin{itemize}
\item Próxima aula
\begin{itemize}
\item Parsing e mônadas.
\end{itemize}
\end{itemize}
\section*{Exercícios}
\label{sec:org5108ca3}

\subsection*{Exercícios}
\label{sec:org185ff97}

\begin{itemize}
\item Estenda a EDSL de HTML para prover suporte a:
\begin{itemize}
\item Listas ordenadas
\item Links
\item Imagens
\end{itemize}
\end{itemize}
\subsection*{Exercícios}
\label{sec:org2b3c6bc}

\begin{itemize}
\item Estenda o tipo de dados de documentos para prover suporte a:
\begin{itemize}
\item Listas ordenadas
\item Links
\item Imagens
\end{itemize}
\end{itemize}
\subsection*{Exercícios}
\label{sec:orgf7b2626}

\begin{itemize}
\item Adapte o processo de compilação para considerar esses novos recursos.

\item Entrega via Github Classroom.
\end{itemize}
\end{document}
