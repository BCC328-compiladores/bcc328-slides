% Intended LaTeX compiler: pdflatex
\documentclass[11pt]{article}
\usepackage[utf8]{inputenc}
\usepackage[T1]{fontenc}
\usepackage{graphicx}
\usepackage{longtable}
\usepackage{wrapfig}
\usepackage{rotating}
\usepackage[normalem]{ulem}
\usepackage{amsmath}
\usepackage{amssymb}
\usepackage{capt-of}
\usepackage{hyperref}
\author{Construção de compiladores I}
\date{}
\title{Análise SLR}
\hypersetup{
 pdfauthor={Construção de compiladores I},
 pdftitle={Análise SLR},
 pdfkeywords={},
 pdfsubject={},
 pdfcreator={Emacs 28.2 (Org mode 9.7)}, 
 pdflang={English}}
\begin{document}

\maketitle
\section*{Objetivos}
\label{sec:org5b707d9}

\subsection*{Objetivos}
\label{sec:org5b3ef79}

\begin{itemize}
\item Apresentar o algoritmo de análise sintática SLR.
\end{itemize}
\section*{Introdução}
\label{sec:org84ccf9f}

\subsection*{Introdução}
\label{sec:org00f7c7e}

\begin{itemize}
\item Nas aulas anteriores, vimos o as funções de fechamento e goto,
utilizadas para a construção de itens LR(0), do autômato LR(0)
e do algoritmo de análise sintática LR(0).
\end{itemize}
\subsection*{Introdução}
\label{sec:orgdc28dc4}

\begin{itemize}
\item Nesta aula, vamos conhecer outro algoritmo de análise
ascendente: SLR

\item O algoritmo SLR consiste de uma pequena modificação na construção
de tabela usada pelo algoritmo LR(0).
\end{itemize}
\section*{Construção da tabela SLR}
\label{sec:org6d13c48}

\subsection*{Construção da tabela SLR}
\label{sec:org74dfa0e}

\begin{itemize}
\item Seja \(G\) a gramática original. Estenda \(G\) com uma nova variável inicial.
Chamaremos essa nova gramática de \(G'\).

\item Calcule follow(\(A\)) para cada não terminal \(A\) de \(G'\).
\end{itemize}
\subsection*{Construção da tabela SLR}
\label{sec:orgfd36233}

\begin{itemize}
\item Construção do AFD LR(0).

\item Construção da tabela SLR.
\end{itemize}
\subsection*{Construção da tabela SLR}
\label{sec:org5c6d1e0}

\begin{itemize}
\item Construa o conjunto \(C=\{I_1,...,I_n\}\) de itens canônicos para a
gramática \(G'\).

\item Cada item \(I_i\) produz o estado \(i\). As ações da tabela são
determinadas como se segue.
\end{itemize}
\subsection*{Construção da tabela SLR}
\label{sec:org2ead519}

\begin{itemize}
\item Se \(A \to \alpha .a\beta \in I_i\) e \(goto(I_i,a) = I_j\), marque a
entrada A[i,a] = shift j.
\end{itemize}
\subsection*{Construção da tabela SLR}
\label{sec:orged7e2a6}

\begin{itemize}
\item Se \(A \to \alpha. \in I_i\), marque A[i,a] = reduce \(A \to \alpha\),
para todo \(a\in follow(A) - \{S'\}\).
\end{itemize}
\subsection*{Construção da tabela SLR}
\label{sec:org904d778}

\begin{itemize}
\item Se \(S' \to S. \in I_i\), marque A[i,\$] = accept.
\end{itemize}
\subsection*{Construção da tabela SLR}
\label{sec:org4001766}

\begin{itemize}
\item Se \(goto(I_i,a) = I_j\), marque G[i,a] = goto j.

\item Qualquer entrada não marcada, são consideradas como rejeitar.
\end{itemize}
\section*{Exemplo}
\label{sec:orge9045d4}

\subsection*{Exemplo}
\label{sec:org7dc4d22}

\begin{itemize}
\item Vamos considerar a gramática de exemplo:
\end{itemize}

\begin{array}{lcl}
  S  & \to & \textbf{(}L\textbf{)}\,|\, \textbf{x}\\
  L  & \to & L\,\textbf{,}\,S\,|\,S\\
\end{array}
\subsection*{Exemplo}
\label{sec:orgfc90e0a}

\begin{itemize}
\item Inicialmente, vamos criar uma nova variável inicial.
\end{itemize}

\begin{array}{lcl}
  S' & \to & S\textbf{.}\\
  S  & \to & \textbf{(}L\textbf{)}\,|\, \textbf{x}\\
  L  & \to & L\,\textbf{,}\,S\,|\,S\\
\end{array}
\subsection*{Exemplo}
\label{sec:orgfb0c5d8}

\begin{itemize}
\item Vamos construir o AFD LR(0).
\end{itemize}
\subsection*{Exemplo}
\label{sec:org2c68545}

\begin{itemize}
\item Inicializando \(C \leftarrow closure(\{S'\to \textbf{.}S\})\).
\begin{itemize}
\item Chamaremos esse conjunto de \(I_1\)
\end{itemize}
\end{itemize}

\begin{array}{rll}
\{ & S \to \textbf{.}S &, \\
   & S \to \textbf{.(}L\textbf{)} & , \\
   & S \to \textbf{.x} & \}\\
\end{array}
\subsection*{Exemplo}
\label{sec:org640023f}

\begin{itemize}
\item Processando \(goto(I_1,x)\):
\begin{itemize}
\item Criando um estado \(I_2\)
\end{itemize}
\end{itemize}

\begin{array}{lcl}
S & \to & \textbf{x.}\\
\end{array}
\subsection*{Exemplo}
\label{sec:org7ee9d1e}

\begin{itemize}
\item Atualizando arestas:
\end{itemize}

\begin{array}{l}
E = \{(I_1,x,I_2)\}
\end{array}
\subsection*{Exemplo}
\label{sec:orgd3c4db5}

\begin{itemize}
\item Processando \(goto(I_1,()\):
\begin{itemize}
\item Criando um estado \(I_3\)
\end{itemize}
\end{itemize}

\begin{array}{lcl}
S & \to & \textbf{(.}L\textbf{)}\\
L & \to & \textbf{.}L\textbf{,}S\\
L & \to & \textbf{.}S\\
S & \to & \textbf{.(}L\textbf{)}\\
S & \to & \textbf{.x}\\
\end{array}
\subsection*{Exemplo}
\label{sec:org0ec3477}

\begin{itemize}
\item Atualizando arestas:
\end{itemize}

\begin{array}{l}
E = \{(I_1,x,I_2),(I_1,(,I_3)\}
\end{array}
\subsection*{Exemplo}
\label{sec:org6f5c886}

\begin{itemize}
\item Processando \(goto(I_1,S)\):
\begin{itemize}
\item Criando um estado \(I_4\)
\end{itemize}
\end{itemize}

\begin{array}{lcl}
S' & \to & S.
\end{array}
\subsection*{Exemplo}
\label{sec:org1b184e2}

\begin{itemize}
\item Atualizando arestas:
\end{itemize}

\begin{array}{l}
E = \{(I_1,x,I_2),(I_1,(,I_3),(I_1,S,I_4)\}
\end{array}
\subsection*{Exemplo}
\label{sec:org483e738}

\begin{itemize}
\item Com isso, concluímos as transições sobre \(I_1\).
\end{itemize}
\subsection*{Exemplo}
\label{sec:org0bc4378}

\begin{itemize}
\item Agora, vamos considerar o conjunto \(I_2\):
\end{itemize}

\begin{array}{lcl}
S & \to & \textbf{x.}\\
\end{array}
\subsection*{Exemplo}
\label{sec:orga0ce101}

\begin{itemize}
\item Nenhuma transição pode ser construída a partir de
\end{itemize}

\begin{array}{lcl}
S & \to & \textbf{x.}\\
\end{array}
\subsection*{Exemplo}
\label{sec:orgf7bede6}

\begin{itemize}
\item Agora, vamos considerar o conjunto \(I_3\):
\end{itemize}

\begin{array}{lcl}
S & \to & \textbf{(.}L\textbf{)}\\
L & \to & \textbf{.}L\textbf{,}S\\
L & \to & \textbf{.}S\\
S & \to & \textbf{.(}L\textbf{)}\\
S & \to & \textbf{.x}\\
\end{array}
\subsection*{Exemplo}
\label{sec:orgc8b8242}

\begin{itemize}
\item Calculando \(goto(I_3,x)\)

\item Única produção a ser considerada:
\end{itemize}

\begin{array}{lcl}
S & \to & \textbf{.x}\\
\end{array}
\subsection*{Exemplo}
\label{sec:org32263f4}

\begin{itemize}
\item Atualizando arestas:
\end{itemize}

\begin{array}{lcl}
E & = & \{(I_1,x,I_2),(I_1,(,I_3),(I_1,S,I_4), \\
  &   &   (I_3,x,I_2) \}\\
\end{array}
\subsection*{Exemplo}
\label{sec:org41d9c66}

\begin{itemize}
\item Logo, obtemos o estado \(I_2\):
\end{itemize}

\begin{array}{lcl}
S & \to & \textbf{x.}\\
\end{array}
\subsection*{Exemplo}
\label{sec:orgfaaf155}

\begin{itemize}
\item Calculando \(goto(I_3,()\).
\item Produção base
\end{itemize}

\begin{array}{lcl}
S & \to & .(L)
\end{array}
\subsection*{Exemplo}
\label{sec:org165c748}

\begin{itemize}
\item Calculando o \(closure(\{S \to (.L)\}\).
\begin{itemize}
\item Incluindo produções \(L\)
\end{itemize}
\end{itemize}

\begin{array}{lcl}
S & \to & (.L)\\
L & \to & \textbf{.}L\textbf{,}S\\
L & \to & \textbf{.}S\\
\end{array}
\subsection*{Exemplo}
\label{sec:orga3d70a5}

\begin{itemize}
\item Calculando o \(closure(\{S \to (.L)\}\).
\begin{itemize}
\item Incluindo produções \(S\)
\item Estado \(I_3\)
\end{itemize}
\end{itemize}

\begin{array}{lcl}
S & \to & (.L)\\
S & \to & .x\\
L & \to & \textbf{.}L\textbf{,}S\\
L & \to & \textbf{.}S\\
S & \to & \textbf{.(}L\textbf{)}\\
\end{array}
\subsection*{Exemplo}
\label{sec:org606bda0}

\begin{itemize}
\item Atualizando arestas:
\end{itemize}

\begin{array}{lcl}
E & = & \{(I_1,x,I_2),(I_1,(,I_3),(I_1,S,I_4), \\
  &   &   (I_3,x,I_2),(I_3,(,I_3) \}\\
\end{array}
\subsection*{Exemplo}
\label{sec:org3e4df22}

\begin{itemize}
\item Calculando \(goto(I_3,L)\)
\begin{itemize}
\item Vamos chamar esse estado de \(I_5\)
\end{itemize}
\end{itemize}

\begin{array}{lcl}
S & \to & (L.)\\
S & \to & L.,S\\
\end{array}
\subsection*{Exemplo}
\label{sec:orgd4eb9f7}

\begin{itemize}
\item Atualizando arestas:
\end{itemize}

\begin{array}{lcl}
E & = & \{(I_1,x,I_2),(I_1,(,I_3),(I_1,S,I_4), \\
  &   &   (I_3,x,I_2),(I_3,(,I_3),(I_3,L,I_5)\} \\
\end{array}
\subsection*{Exemplo}
\label{sec:org7281b78}

\begin{itemize}
\item Calculando \(goto(I_3,S)\):
\begin{itemize}
\item Chamaremos esse estado de \(I_6\)
\end{itemize}
\end{itemize}

\begin{array}{lcl}
L & \to & S.\\
\end{array}
\subsection*{Exemplo}
\label{sec:org997d5f4}

\begin{itemize}
\item Atualizando arestas:
\end{itemize}

\begin{array}{lcl}
E & = & \{(I_1,x,I_2),(I_1,(,I_3),(I_1,S,I_4), \\
  &   &   (I_3,x,I_2),(I_3,(,I_3),(I_3,L,I_5), \\
  &   &   (I_3,S,I_6)\}
\end{array}
\subsection*{Exemplo}
\label{sec:orga667934}

\begin{itemize}
\item Agora, vamos considerar o estado \(I_4\)
\begin{itemize}
\item Não há transições possíveis.
\end{itemize}
\end{itemize}

\begin{array}{lcl}
S' & \to & S.
\end{array}
\subsection*{Exemplo}
\label{sec:orgfac539e}

\begin{itemize}
\item Agora, vamos considerar o estado \(I_5\)
\end{itemize}

\begin{array}{lcl}
S & \to & (L.)\\
S & \to & L.,S\\
\end{array}
\subsection*{Exemplo}
\label{sec:org2580d40}

\begin{itemize}
\item Calculando \(goto(I_5,))\)
\begin{itemize}
\item Chamaremos esse estado de \(I_7\).
\end{itemize}
\end{itemize}

\begin{array}{lcl}
S & \to & (L).\\
\end{array}
\subsection*{Exemplo}
\label{sec:org4e97a37}

\begin{itemize}
\item Atualizando arestas:
\end{itemize}

\begin{array}{lcl}
E & = & \{(I_1,x,I_2),(I_1,(,I_3),(I_1,S,I_4), \\
  &   &   (I_3,x,I_2),(I_3,(,I_3),(I_3,L,I_5), \\
  &   &   (I_3,S,I_6),(I_5,),I_7)\}
\end{array}
\subsection*{Exemplo}
\label{sec:orgcf40877}

\begin{itemize}
\item Calculando \(goto(I_5, ,)\)
\begin{itemize}
\item Chamaremos esse estado de \(I_8\)
\end{itemize}

\item Produção base
\end{itemize}

\begin{array}{lcl}
S & \to & L,.S\\
\end{array}
\subsection*{Exemplo}
\label{sec:orgb3dcf7d}

\begin{itemize}
\item Calculando \(closure(\{S\to L,.S\})\):
\begin{itemize}
\item Chamaremos esse estado de \(I_8\)
\end{itemize}
\end{itemize}

\begin{array}{lcl}
S & \to & L,.S\\
S & \to & .(L)\\
S & \to & .x\\
\end{array}
\subsection*{Exemplo}
\label{sec:org410734b}

\begin{itemize}
\item Atualizando arestas:
\end{itemize}

\begin{array}{lcl}
E & = & \{(I_1,x,I_2),(I_1,(,I_3),(I_1,S,I_4), \\
  &   &   (I_3,x,I_2),(I_3,(,I_3),(I_3,L,I_5), \\
  &   &   (I_3,S,I_6),(I_5,),I_7), (I_5,,,I_8)\}
\end{array}
\subsection*{Exemplo}
\label{sec:org21a5d6c}

\begin{itemize}
\item Agora, vamos considerar o estado \(I_6\):
\begin{itemize}
\item Não há transições possíveis.
\end{itemize}
\end{itemize}

\begin{array}{lcl}
L & \to & S.\\
\end{array}
\subsection*{Exemplo}
\label{sec:org38baebb}

\begin{itemize}
\item Agora vamos considerar o estado \(I_7\):
\begin{itemize}
\item Não há transições possíveis.
\end{itemize}
\end{itemize}

\begin{array}{lcl}
S & \to & (L).\\
\end{array}
\subsection*{Exemplo}
\label{sec:org4e294a3}

\begin{itemize}
\item Agora vamos consderar o estado \(I_8\):
\end{itemize}

\begin{array}{lcl}
S & \to & L,.S\\
S & \to & .(L)\\
S & \to & .x\\
\end{array}
\subsection*{Exemplo}
\label{sec:orge914c41}

\begin{itemize}
\item Calculando \(goto(I_8,x)\)

\item Produção base: \(S \to .x\)

\item Resultado: estado \(I_2\)
\end{itemize}
\subsection*{Exemplo}
\label{sec:orge1fccef}

\begin{itemize}
\item Atualizando arestas:
\end{itemize}

\begin{array}{lcl}
E & = & \{(I_1,x,I_2),(I_1,(,I_3),(I_1,S,I_4), \\
  &   &   (I_3,x,I_2),(I_3,(,I_3),(I_3,L,I_5), \\
  &   &   (I_3,S,I_6),(I_5,),I_7), (I_5,,,I_8), \\
  &   &   (I_8,x,I_2)\}
\end{array}
\subsection*{Exemplo}
\label{sec:orgd0db717}

\begin{itemize}
\item Calculando \(goto(I_8,()\)

\item Produção base: \(S \to (.S)\)

\item Resultado: estado \(I_3\).
\end{itemize}
\subsection*{Exemplo}
\label{sec:org5547ed1}

\begin{itemize}
\item Atualizando arestas:
\end{itemize}

\begin{array}{lcl}
E & = & \{(I_1,x,I_2),(I_1,(,I_3),(I_1,S,I_4), \\
  &   &   (I_3,x,I_2),(I_3,(,I_3),(I_3,L,I_5), \\
  &   &   (I_3,S,I_6),(I_5,),I_7), (I_5,,,I_8), \\
  &   &   (I_8,x,I_2),(I_8,(,I_3))\}
\end{array}
\subsection*{Exemplo}
\label{sec:orgea60184}

\begin{itemize}
\item Calculando \(goto(I_8,S)\)
\begin{itemize}
\item Vamos chamar esse estado de \(I_9\).
\end{itemize}

\item Produção base: S \(\to\) L,S.
\end{itemize}

\begin{array}{lcl}
S & \to & L,S.
\end{array}
\subsection*{Exemplo}
\label{sec:org42f6a6a}

\begin{itemize}
\item Atualizando arestas:
\end{itemize}

\begin{array}{lcl}
E & = & \{(I_1,x,I_2),(I_1,(,I_3),(I_1,S,I_4), \\
  &   &   (I_3,x,I_2),(I_3,(,I_3),(I_3,L,I_5), \\
  &   &   (I_3,S,I_6),(I_5,),I_7), (I_5,,,I_8), \\
  &   &   (I_8,x,I_2),(I_8,(,I_3)), (I_8,S,I_9)\}
\end{array}
\subsection*{Exemplo}
\label{sec:org1faa480}

\begin{itemize}
\item Agora vamos considerar o estado \(I_9\)
\begin{itemize}
\item Não há transições possíveis.
\end{itemize}
\end{itemize}

\begin{array}{lcl}
S & \to & L,S.
\end{array}
\subsection*{Exemplo}
\label{sec:org61df5cf}

\begin{itemize}
\item Como não há modificações, o algoritmo termina

\item Agora, temos o AFD LR(0) para a gramática.
\end{itemize}
\subsection*{Exemplo}
\label{sec:org3232c4e}

\begin{itemize}
\item Desenho do AFD LR(0) para a gramática na lousa.
\end{itemize}
\subsection*{Exemplo}
\label{sec:org73157a3}

\begin{itemize}
\item Construção dos conjuntos follow(A)
\end{itemize}
\subsection*{Exemplo}
\label{sec:org04483eb}

\begin{itemize}
\item Primeiro calculando os conjuntos first.
\begin{itemize}
\item first(S') = first(S) = first((L)) \(\cup\) first(x) = \{(,x\}.
\item first(L) = first(S) = \{(,x\}.
\end{itemize}
\end{itemize}
\subsection*{Exemplo}
\label{sec:org6e4b8c7}

\begin{itemize}
\item Calculando follow:
\begin{itemize}
\item follow(S') = \{\$,(, ,\}
\item follow(S) = \{ ), ,\}
\item follow(L) = \{ ), ,\}
\end{itemize}
\end{itemize}
\subsection*{Exemplo}
\label{sec:orgb0fd8b4}

\begin{itemize}
\item Agora vamos construir a tabela de análise.

\item Primeiro, considerando o estado \(I_1\).
\end{itemize}
\subsection*{Exemplo}
\label{sec:orgc463154}

\begin{itemize}
\item Produção \(S \to x\):
\begin{itemize}
\item Como goto(\(I_1\),x) = \(I_2\), temos que A[1,x] = shift 2.
\end{itemize}
\end{itemize}
\subsection*{Exemplo}
\label{sec:orgebe535d}

\begin{itemize}
\item Produção \(S \to (L)\):
\begin{itemize}
\item Como goto(\(I_1\),() = \(I_3\), temos que A[1,(] = shift 3.
\end{itemize}
\end{itemize}
\subsection*{Exemplo}
\label{sec:org4f48c18}

\begin{itemize}
\item Como goto(\(I_1\), S) = \(I_4\), temos que G[1,S] = goto 4.
\end{itemize}
\subsection*{Exemplo}
\label{sec:orga00a250}

\begin{itemize}
\item Considerando o estado \(I_2\).

\item Como S \(\to\) x. e follow(S) = \{), ,\}, temos que
\begin{itemize}
\item A[2,)] = A[2, ,] = reduce \(S \to x\).
\end{itemize}
\end{itemize}
\subsection*{Exemplo}
\label{sec:orgda56097}

\begin{itemize}
\item Considerando o estado \(I_3\).
\begin{itemize}
\item Como goto(\(I_3\),x) = \(I_2\), temos que A[3,x] = shift 2.
\item Como goto(\(I_3,(\)) = \(I_3\), temos que A[3,)] = shift 3.
\end{itemize}
\end{itemize}
\subsection*{Exemplo}
\label{sec:org52592b6}

\begin{itemize}
\item Como goto(\(I_3\),L) = \(I_5\), temos que G[3,L] = goto 5.
\item Como goto(\(I_3\),S) = \(I_6\), temos que G[3,S] = goto 6.
\end{itemize}
\subsection*{Exemplo}
\label{sec:org35f1ae7}

\begin{itemize}
\item Considerando o estado \(I_4\).
\begin{itemize}
\item Como S' \(\to\) S., temos que A[4,\$] = accept.
\end{itemize}
\end{itemize}
\subsection*{Exemplo}
\label{sec:orge904868}

\begin{itemize}
\item Considerando o estado \(I_5\).
\begin{itemize}
\item Como goto(\(I_5\),)) = \(I_7\), temos que A[5,)] = shift 7.
\item Como goto(\(I_5\), ,) = \(I_8\), temos que A[5, ,] = shift 8.
\end{itemize}
\end{itemize}
\subsection*{Exemplo}
\label{sec:org28dcfb9}

\begin{itemize}
\item Considerando o estado \(I_6\).
\begin{itemize}
\item Como L \(\to\) S. e follow(L) = \{), ,\}, temos que:
\begin{itemize}
\item A[6,)] = A[6, ,] = reduce \(L \to S\).
\end{itemize}
\end{itemize}
\end{itemize}
\subsection*{Exemplo}
\label{sec:orgead9341}

\begin{itemize}
\item Considerando o estado \(I_7\).
\begin{itemize}
\item Como S \(\to\) (L) e follow(S) = \{), ,\}, temos que:
\begin{itemize}
\item A[7,)] = A[7, ,] = reduce \(S \to (L)\)
\end{itemize}
\end{itemize}
\end{itemize}
\subsection*{Exemplo}
\label{sec:org60b0c29}

\begin{itemize}
\item Considerando o estado \(I_8\).
\begin{itemize}
\item Como goto(\(I_8\), () = \(I_3\), temos que A[8,(] = shift 3
\item Como goto(\(I_8\), x) = \(I_2\), temos que A[8,x] = shift 2
\end{itemize}
\end{itemize}
\subsection*{Exemplo}
\label{sec:org128be5d}

\begin{itemize}
\item Como goto(\(I_8\),S) = \(I_9\), temos que G[8,S] = goto 9.
\end{itemize}
\subsection*{Exemplo}
\label{sec:org89aff83}

\begin{itemize}
\item Considerando o estado \(I_9\).
\begin{itemize}
\item Como L  \(\to\) L,S. e follow(L) = \{), ,\}, temos que:
\begin{itemize}
\item A[9,)] = A[9, ,] = reduce \(L \to L, S\).
\end{itemize}
\end{itemize}
\end{itemize}
\subsection*{Exemplo}
\label{sec:org9d3e0ef}

\begin{itemize}
\item Desenho da tabela na lousa.

\item Uso da tabela para análise sintática de (x,x).
\end{itemize}
\section*{Concluindo}
\label{sec:orgd575f9f}

\subsection*{Concluindo}
\label{sec:org746ef5d}

\begin{itemize}
\item Nesta aula apresentamos a construção de tabelas SLR.

\item Próxima aula: Analisadores sintáticos LR(1).
\end{itemize}
\section*{Exercícios}
\label{sec:org18eb99b}

\subsection*{Exercícios}
\label{sec:orgf1a77de}

\begin{itemize}
\item Determine se a seguinte gramática possui conflitos,
utilizando o algoritmo de construção de tabelas SLR.
\end{itemize}

\begin{array}{lcl}
E & \to & T \textbf{+} E\,|\,T\\
T & \to & \textbf{x}\\
\end{array}
\end{document}
